%! Author = mboehme
%! Date = 21.02.2023

% Preamble

% Document

\section{Einleitung}\label{sec:einleitung}
Die Praxisarbeit befasst sich mit der Ressourcen- und Terminplanung in Office365 mithilfe von Microsoft Graph API. Die Arbeit ist in drei Teile gegliedert.
Im ersten Teil wird die Theorie der Ressourcen- und Terminplanung in Office365 mit Hilfe von Microsoft Graph API erläutert.
Im zweiten Teil wird die praktische Umsetzung der Theorie beschrieben. Im dritten Teil wird die Arbeit abschließend bewertet.
    \newline
    \subsection{Fragestellungen der Arbeit}\label{subsec:fragestellungen-der-arbeit}
Die Fragestellungen der Arbeit lauten:
    \begin{itemize}
        \item Wie funktioniert die Ressourcen- und Terminplanung in Office365 mithilfe von Microsoft Graph API?
        \item Wie kann die Ressourcen- und Terminplanung in Office365 mithilfe von Microsoft Graph API praktisch umgesetzt werden?
        \item Wie kann die Arbeit abschließend bewertet werden?
    \end{itemize}
    \subsection{Ziele der Arbeit}\label{subsec:ziele-der-arbeit}
Die Ziele der Arbeit sind deshalb wie folgt:
    \begin{itemize}
        \item Die Theorie der Ressourcen- und Terminplanung in Office365 mithilfe von Microsoft Graph API zu erläutern.
        \item Die praktische Umsetzung der Theorie anhand eines Kundenauftrags zu beschreiben.
        \item Die Arbeit abschließend zu bewerten.
    \end{itemize}
    \subsection{Ergebnisse der Arbeit}\label{subsec:ergebnisse-der-arbeit}
Die Arbeit hat alle Ziele erreicht.
    Die Microsoft Graph API wurde erfolgreich eingesetzt und die Ressourcen- und Terminplanung in Office365 mithilfe von Microsoft Graph API wurde erfolgreich umgesetzt als SPA (Single Page Application).
    Die Total Blocking Time (TBT) der Applikation liegt, aufgrund von hoher Komplexität, bei ca. 500ms.
    Nachdem die Seite jedoch erstmal geladen wurde, dauert beispielsweise das Öffnen des Buchungsdialogs nur 2ms, welche die hauptsächliche Interaktion des Anwenders mit der Seite darstellt.
    \newline
\newpage

