%! Author = mboehme
%! Date = 21.02.2023

% Preamble

% Document

\section{Einleitung}\label{sec:einleitung}
Die Praxisarbeit befasst sich mit der Ressourcen- und Terminplanung in Office365 mithilfe von Microsoft Graph API~\cite{microsoftGraphApi}.
Die Arbeit ist in drei Teile gegliedert.
Im ersten Teil wird die Theorie der Ressourcen- und Terminplanung in Office365 mithilfe von Microsoft Graph API erläutert.
Im zweiten Teil wird die praktische Umsetzung der Theorie beschrieben.
Im dritten Teil wird die Arbeit abschließend bewertet.
    \newline
    \subsection{Fragestellungen der Arbeit}\label{subsec:fragestellungen-der-arbeit}
Die Fragestellungen der Arbeit lauten:
    \begin{itemize}
        \item Wie funktioniert die Ressourcen- und Terminplanung in Office365 mithilfe von Microsoft Graph API?
        \item Wie kann die Ressourcen- und Terminplanung in Office365 mithilfe von Microsoft Graph API praktisch umgesetzt werden?
        \item Wie kann die Arbeit abschließend bewertet werden?
    \end{itemize}
Konkret bedeutet dies:
\newline
Inwiefern und wie sinnvoll, ist der Einsatz von Microsoft Graph API für die Ressourcen- und Terminplanung in Office365, im Vergleich zu anderen APIs.
Dabei sollte berücksichtigt werden, dass dies auf den Kundenauftrag bezogen ist.
Wie sollte sowas umgesetzt werden und welche Aspekte der Microsoft Graph API sind dafür relevant?
Letztenendes soll herausgefunden werden, wie man die Arbeit quantitativ und qualitativ bewerten kann.
    \subsection{Ziele der Arbeit}\label{subsec:ziele-der-arbeit}
Die Ziele der Arbeit sind wie folgt:
    \begin{itemize}
        \item Die Theorie der Ressourcen- und Terminplanung in Office365 mithilfe von Microsoft Graph API zu erläutern.
        \item Die praktische Umsetzung der Theorie anhand eines Kundenauftrags zu beschreiben.
        \item Die Arbeit abschließend zu bewerten.
    \end{itemize}
Für die Ziele bedeutet dies, dass die Fragestellungen der Arbeit beantwortet werden müssen.
Sowohl die Theorie als auch die praktische Umsetzung der Theorie, müssen in der Arbeit beschrieben werden.
Es muss immer wieder auf die Kundenanforderungen zurückgegriffen werden.

    \subsection{Ergebnisse der Arbeit}\label{subsec:ergebnisse-der-arbeit}
Die Arbeit hat alle Ziele erreicht.
    \newglossaryentry{SPA}{name=Single Page Application (SPA), description={Eine Single Page Application (SPA) ist eine Webanwendung, die nur eine HTML-Seite besitzt.
    Diese Seite wird beim Laden der Anwendung geladen und bleibt während der gesamten Nutzung der Anwendung bestehen.}, first={Single Page Application (SPA)}, text={SPA}, short=SPA}
%\newglossaryentry{SPA}{name=SPA, description={Eine Single Page Application (SPA) ist eine Webanwendung, die nur eine HTML-Seite besitzt.}}
%\newacronym{SPA}{SPA}{Single Page Application}
    Die Microsoft Graph API ist erfolgreich eingesetzt und die Ressourcen- und Terminplanung in Office365, mithilfe von der Microsoft Graph API, wurde erfolgreich als~\gls{SPA} umgesetzt.
\newglossaryentry{TBT}{name={Total-Blocking-Time (TBT)}, description= {Die Total-Blocking-Time (TBT) ist eine Metrik, die die Gesamtzeit misst, die eine Seite blockiert wird, bis sie interaktiv ist.},
first={Total-Blocking-Time (TBT)}, text={TBT}}
Die~\gls{TBT} der Applikation liegt, aufgrund von hoher Komplexität, bei ca. 500ms.
    Nachdem die Seite jedoch erstmal geladen wurde, dauert beispielsweise das Öffnen des Buchungsdialogs nur 2ms, welche die hauptsächliche Interaktion des Anwenders mit der Seite darstellt.
    \newline
Die detaillierten Aspekte Ergebnisse der Arbeit sind in Kapitel~\ref{sec:Theoretische Grundlagen} und Kapitel~\ref{sec:technische-umsetzung} beschrieben.
Das ausformulierte Ergebnis und die Schlussfolgerung dieser, sind in Kapitel~\ref{sec:ergebnis} erläutert.
\newpage

