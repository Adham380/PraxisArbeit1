%! Author = Adham Aijou
%! Date = 23.02.2023
\subsection{Systemschnittstellen}\label{subsec:systemschnittstellen}
Hier wird ein Mal die benötigte Kommunikation mit der Microsoft Graph API als Systemschnittstelle, tabellarisch erläutert.
\newline
%make tabularx table with two columns, the columns should take up the whole width of the page and adjust to the width of the page. Two columns.
\footnotesize
\begin{tabularx}{\textwidth}{|X|X|}
    \hline
%\begin{tabularx}{\textwidth\footnotesize}{|l*{2}{|X}}
%    \multicolumn{2}{|c|}{\textbf{Microsoft Graph API}}\\
    \caption{Termin buchen}
    \label{tab:TerminBuchen}
 Kurzbeschreibung: & Im UI kann der User sich mit der Ressource einloggen, die für die Zukunft als Repräsentant für die reelle Ressource (z.B einen Raum) gelten soll.
    Daraufhin gilt dieser User immer als Teilnehmer der Buchungen und kann in seinen verfügbaren Zeiten gebucht werden.
    Basierend auf diesen Zeiten, bekommt der User verfügbare timeslots zur Buchung angezeigt.
    Auch die Dauer des Termins soll einstellbar sein.
    Es gibt zudem einen optionalen Betreff für den Termin.
    Der gebuchte Termin reflektiert sich dann im UI in der Liste der für den Tag schon gebuchten Termine und den Terminen vom nächsten Tag, für die Ressource.
    Zudem soll der Nutzer, anhand eines farblichen UIs (siehe Abschnitt~\ref{subsec:anforderungen}) sehen können ob innerhalb der nächsten 15 Minuten ein Termin für die Ressource ansteht oder ob einer schon am Passieren ist.

    REST API Anfragen werden dann an die Microsoft Graph API geschickt, um diese Wünsche zu kommunizieren.\\
    \hline
    Akteure: & Mitarbeiter\\
    \hline
    Häufigkeit: & Variabel pro Ressource, Firma und Anzahl Ressourcen\\
    \hline
    Komplexität: & Mittlere bis hohe Komplexität\\
    \hline
    Vorbedingungen: & \begin{enumerate}
                          \item User muss eingeloggt sein (Nutzerdaten können nicht gesehen oder eingespeichert werden) und Zugriffsrechte akzeptieren.
                          \item Optionale Teilnehmer
                          \item Termindauer
                          \item Verfügbare Termine
                          \item Terminbetreff
                          \item Zeitzone
                          \item Für die Microsoft Graph API kommt noch der Access Token / Client ID von unserer Azure App zur Authentifizierung dazu
    \end{enumerate}\\
    \hline
    Ausgaben: & \begin{enumerate}
                    \item Buchung erfolgreich oder nicht
                    \item Neue verfügbare Termine
                    \item Der neue Termin wird in der Liste der gebuchten Termine angezeigt
                    \item Falls die Buchung nicht geklappt haben sollte, wird der User darüber informiert
    \end{enumerate}\\
    \hline
    Use Case: &~\ref{tab:Terminbuchung}\\
    \hline
Schnittstellen: & \begin{enumerate}
                        \item Microsoft Graph API
                        \item User Interface
                        \item Backend
                      \end{enumerate}\\
    \hline
    Aufgerufene Aktionen: & Notwendige: \begin{enumerate}
                                        \item Termin buchen
                                        \item Termin finden
                                            \end{enumerate}
            \linebreak Optionale: \begin{enumerate}
                                      \item Teilnehmer hinzufügen
                                      \item Terminbetreff hinzufügen/ändern
                                        \item Termindauer ändern
                                      \end{enumerate}\\
    \hline
\end{tabularx}
\normalsize
\pagebreak