%! Author = Adham Aijou
%! Date = 23.02.2023
\subsection{Systemschnittstellen}\label{subsec:systemschnittstellen}
%Bitte "Umgangssprache" entfernen
Hier wird ein Mal die benötigte Kommunikation mit der Microsoft Graph API als Systemschnittstelle tabellarisch erläutert.
\newline
%make tabularx table with two columns, the columns should take up the whole width of the page and adjust to the width of the page. Two columns.
\footnotesize
\begin{tabularx}{\textwidth}{|X|X|}
    \hline
%\begin{tabularx}{\textwidth\footnotesize}{|l*{2}{|X}}
%    \multicolumn{2}{|c|}{\textbf{Microsoft Graph API}}\\
    \caption{Termin buchen}
    \label{tab:TerminBuchen}
%Die Kurzbeschreibung ist zu lang und hat nichts mit der Systemschnittstelle zu tun. Bitte in Use Case verschieben
%Im Fließtext erläutern
 Kurzbeschreibung: & In der UI kann der User sich mit Konto der Ressource anmelden, die für die Zukunft als Repräsentant für die reelle Ressource (z.B einem Raum) gelten soll.
    Daraufhin gilt dieser Raum immer als Teilnehmer der Buchungen und kann in seinen verfügbaren Zeiten gebucht werden.
    Basierend auf diesen Zeiten, bekommt der Benutzer verfügbare timeslots zur Buchung angezeigt.
    Die Dauer des Termins soll einstellbar sein.
    Es gibt zudem einen optionalen Betreff für den Termin.
    Der gebuchte Termin wird in der Terminübersicht, für den aktuellen und folgenden Tag, angezeigt.
    Zudem soll der Nutzer anhand farblicher Indikatoren (siehe Abschnitt~\ref{subsec:anforderungen}) beurteilen können, ob die Ressource verfügbar ist.
    \newglossaryentry{REST API}{name=REST API, description={REST API ist eine Abkürzung für Representational State Transfer Application Programming Interface. REST ist ein Architekturstil, der die Kommunikation zwischen verschiedenen Systemen ermöglicht. REST ist ein Architekturstil, der die Kommunikation zwischen verschiedenen Systemen ermöglicht.}}
    ~\gls{REST API} Anfragen werden dann an die Microsoft Graph API geschickt, um diese Buchung zu kommunizieren.\\
    \hline
    Akteure: & Mitarbeiter und Software\\
    \hline
%    Ressource besser definieren
    Häufigkeit: & Abhängig von Firmengröße und Besprechungsfrequenz\\
    \hline
    Komplexität: & Mittlere bis hohe Komplexität\\
    \hline
    Vorbedingungen: & \begin{enumerate}
%                          (Nutzerdaten können nicht gesehen oder eingespeichert werden)  woanders
                          \item Der Benutzer muss eingeloggt sein und Zugriffsrechte akzeptiert haben.
                          \item Optionale Teilnehmer
                          \item Termindauer
                          \item Verfügbare Termine
                          \item Terminbetreff
                          \item Zeitzone
                          \item Access Token / Client ID zur Authentifizierung
    \end{enumerate}\\
    \hline
    Ausgaben: & \begin{enumerate}
                    \item Buchung erfolgreich oder nicht
                    \item Neue verfügbare Termine
                    \item Der neue Termin wird in der Liste der gebuchten Termine angezeigt
    \end{enumerate}\\
    \hline
    Use Case: & Siehe ~\ref{tab:Terminbuchung}\\
    \hline
    Ausgeführte Aktionen: & Immer auszuführen: \begin{enumerate}
                                                     \item Termin finden
                                                   \item Termin buchen
                                            \end{enumerate}
            \linebreak Optional auszuführen: \begin{enumerate}
                                      \item Teilnehmer hinzufügen
                                      \item Terminbetreff hinzufügen/ändern
                                        \item Termindauer ändern
                                      \end{enumerate}\\
    \hline
\end{tabularx}
\normalsize
\pagebreak