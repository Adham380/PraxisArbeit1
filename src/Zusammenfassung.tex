%! Author = mboehme
%! Date = 23.02.2023
% Preamble
\section*{Zusammenfassung}\label{sec:Zusammenfassung}
Die Praxisarbeit wurde durch einen Kundenauftrag ins Leben gerufen.
Es sei umständlich für gewisse Kunden für Konferenzräume Termine zu buchen.
Jedes Mal muss nämlich ein Mitarbeiter die Verfügbarkeit eines Konferenzraumes auf einer Applikation mit zu vielen Zwischenschritten prüfen und dann einen Termin buchen.
Dieser Prozess ist sehr zeitaufwendig und kann zu Fehlern führen.

\newline
\newline
%Muss deutlicher werden, dass das keine Analyse sein soll
Daher soll eine Anwendung entwickelt werden, die es ermöglichen soll, dass der Mitarbeiter selbstständig einen Termin buchen oder einsehen kann.
Außerdem soll überprüft werden können, wann Termine für den Raum stattfinden sollen, wenn die Person schon im Gebäude und sich vor dem Raum befindet.
\newline
Bei diesem Kunden sind viele Mitarbeiter nicht aus der gleichen Abteilung oder dem gleichen Subunternehmen und besitzen folglich unterschiedliche Firmennamen und Logos.
Wenn ein Kalender keine detaillierteren Termininformationen darstellt, müsste der Mitarbeiter immer noch nachschauen, ob der Termin für ihn gedacht sei.
Die Buchungen direkt vor dem Raum sind dafür gedacht, dass ein Mitarbeiter den Raum sofort buchen kann.
\newline
%Umschreiben
Sollte dies nicht der Fall sein, müsste der Anwender, im Voraus, an einem anderen Endgerät, welches Outlook oder Teams besitzt, einen detaillierten Termin buchen, damit alle Beteiligten wissen, wann der Termin stattfindet und wer anwesend sein wird.
%Vllt mit Zeile 21 verknüpfen
Dort kann der Anwender dann auch angeben, welche Firma der Gastgeber und welche der Gast sein soll.
