%! Author = mboehme
%! Date = 23.02.2023

% Preamble
\subsection{Zusammenfassung}
Die Praxisarbeit wurde durch einen Kundenauftrag ins Leben gerufen.
Es sei umständlich für gewisse Kunden für Konferenzräume Termine zu buchen.
Jedes Mal muss ein Mitarbeiter des Kunden die Verfügbarkeit eines Konferenzraumes, auf einer Applikation, mit zu vielen Zwischenschritten prüfen und dann einen Termin buchen.
Dieser Prozess ist sehr zeitaufwendig und kann zu Fehlern führen.
Falls beispielsweise 30 Räume zur Verfügung stehen und man nur diesen einen Raum auf Verfügbarkeit prüfen möchte, ist dies einerseits umständlich und vielleicht durch den Administrator eingeschränkt.
Wenn der Administrator dann den Raum für den Kunden freigibt, muss der Mitarbeiter den Raum nochmal auf Verfügbarkeit prüfen, bevor er den Termin bucht und falls zu viele Rechte freigegeben werden, kann es zu Fehlern kommen,
die dann von einem Administrator behoben werden müssen.
Zudem hätten dann Mitarbeiter Rechte, die sie nicht benötigen und könnten außerhalb ihrer Zuständigkeit, Fehler machen.
\newline
Daher soll eine Applikation entwickelt werden, die es ermöglichen soll, dass der Mitarbeiter selbstständig einen Termin buchen, oder einsehen kann, wann ein Termin denn stattfinden soll, wenn er sowieso schon im Gebäude und sich vor dem Raum befindet.
\newline
Bei diesem Kunden sind viele Mitarbeiter nicht aus der gleichen Abteilung oder dem gleichen Subunternehmen.
Alle besitzen ihre eigenen Logos und Namen.
Wenn also ein Termin nur anzeigt, wann er stattfindet, müsste der Mitarbeiter immer noch nachschauen, ob der Termin für ihn gedacht ist.
Die Buchungen direkt vor dem Raum sind jedoch nur dafür gedacht, dass der Mitarbeiter den Raum für sich reserviert, falls er ihn jetzt gerade oder bald benötigt und dann wissen die Teilnehmer auch im Regelfall, dass der Termin für sie gedacht ist.
\newline
Sollte dies nicht der Fall sein, sollte man im Voraus, an einem anderen Endgerät, welches Outlook oder Teams besitzt, einen detaillieren Termin buchen, damit alle Beteiligten wissen, wann der Termin stattfindet und wer anwesend sein wird und dort kann man dann auch angeben, welche Firma der Gastgeber ist und welche der Gast.
