%! Author = mboehme
%! Date = 01.03.2023

\newpage
\section{Ergebnis}\label{sec:ergebnis}
Das Ergebnis sieht wie folgt aus:
\subsection{Visuelle Darstellung}\label{subsec:visuelle-darstellung}
%include two images
\par\vspace{1cm}
    \centering
    \includegraphics[width=0.8\textwidth]{Bilder/Ergebnis}
    \caption{Ergebnis mit nächstem anstehenden Termin}
    \label{fig:Ergebnis mit nächstem anstehenden Termin}
\par\vspace{1cm}
\raggedright
\par\vspace{1cm}
    \centering
    \includegraphics[width=0.8\textwidth]{Bilder/Ergebnis_LaufenderTermin}
    \caption{Ergebnis mit laufendem Termin}
    \label{fig:Ergebnis mit laufendem Termin}
\par\vspace{1cm}
\raggedright
\newline
Oben links im Bild, das Bild des kleinen roten Charakters, ist das Logo des Gastgebers des nächsten, beziehungsweise jetzigen, Termins zu sehen.
Solch ein Logo kann dargestellt werden, indem beim Erstellen des Termins, außerhalb des Tablets, bei Outlook beispielsweise, ein Bild hochgeladen wird, welches im Betreff eine bestimmte Bezeichnung enthält, die hier aus Sicherheitsgründen nicht genannt werden kann.
\newline
Die anderen Logos sind alle Logos vom Gast des Termins.
Diese Logos können an alle Geräte gleichzeitig versendet werden, indem ein spezieller Termin erstellt wird, der nur für die Logos gedacht ist und eine einzigarte ID, sowie Befehle enthält, die dann das Bild, inklusive Firmennamen, in einer lokalen Datenbank abspeichert.
Diese Logos können hinzugefügt, gelöscht oder aktualisiert werden.
Aus Sicherheitsgründen werden die genaueren Befehle der Schnittstelle hier nicht genannt.
\newline
Es wird nur der erste Gast angezeigt, da das so vom Kunden gewünscht wurde.
\newline
Die Uhrzeit wird, immer, in der Zeitzone des Tablets angezeigt.
\newline
\newline
Hier sieht man nochmal das Menü, wo ein Termin gebucht werden kann, welches durch das Drücken des Plus-Symbols aufgerufen wird:
\par\vspace{1cm}
    \centering
    \includegraphics[width=0.8\textwidth]{Bilder/Ergebnis_TerminErstellen_Menue}
    \caption{Menü}
    \label{fig:Menue}
\par\vspace{1cm}
\raggedright
Wie zu sehen ist, ist beispielsweise die selbsterstellte Option \("\)Jetzt\("\) deaktiviert, da das Ende des hypothetischen Termins innerhalb von 15 Minuten vom nächsten anstehenden Termin beginnt.
Diese Pufferzeit ist so mit dem Kunden abgesprochen.
Es werden alle möglichen Termine, innerhalb der Arbeitszeiten, für den Jetzigen und nächsten Tag angezeigt.
Falls z.B. der nächste Tag ein Feiertag ist, werden für den nächsten Tag keine Termine angezeigt.
Auf die anderen Optionen, wie z.B. Pufferzeiten, hat ein Anwender hier wenig Einfluss, da sie von den Einstellungen des Kalenders, des jeweiligen Nutzers, abhängen und es somit in der Verantwortung des Administrators liegt, diese zu ändern.
\newline
\newline
Hier die normale Ansicht nochmal, aber im light-Modus:
\par\vspace{1cm}
    \centering
    \includegraphics[width=0.8\textwidth]{Bilder/Ergebnis_lightMode}
    \caption{Prototyp im light-Modus}
    \label{fig:PrototypLight}
\par\vspace{1cm}
\raggedright
\newline
\newline
\subsection{Performance}\label{subsec:performance}
Die Performance einer \gls{SPA} zu messen ist trotz der geringen Komplexität der Anwendung, nicht simpel.
Auch die gängige Total-Blocking-Time Messung ist nicht zwingend sinnvoll, da die Anwendung nicht für den Nutzer, während der Nutzung, blockiert ist, sondern lediglich die Daten vom Server geladen werden, um anschließend die Nutzung der Anwendung sinnvoll zu ermöglichen.
Daher wurden manuell einige Tests durchgeführt, um die Performance zu messen.
\newline
\newline
\subsubsection{Test 1}\label{subsubsec:test-1}
\newline
\newline
Wie lange braucht die Anwendung, um nach einem Klick auf den Button \("\)+\("\) das Terminerstellungs-Menü zu öffnen?
\gls{Event-Listener}{name={Event-Listener},description={Ein Event-Listener ist ein Objekt, welches auf bestimmte Ereignisse wartet und dann eine Funktion ausführt. Dieses Objekt braucht ein HTML Element, welches es überwachen soll.}}
Die Antwort auf diese Fragestellung wurde gemessen, indem der Event-Listener für den Button \("\)+\("\) registriert ausgab, wann die Taste betätigt und damit Zeit gemessen wurde, die vergeht, bis der \gls{Event-Listener} das Menü darstellt.
Es dauert im Durchschnitt 2ms, bis das Menü angezeigt wird.
\newline
\newline
\subsubsection{Test 2}\label{subsubsec:test-2}
\newline
\newline
Wie viele Bilder pro Sekunde zeigt die Anwendung durchschnittlich an?
Dies wurde mithilfe von LightHouse gemessen.
Die Anwendung zeigt durchschnittlich 60 Bilder pro Sekunde an.
\newline
\newline
\subsubsection{Test 3}\label{subsubsec:test-3}
\newline
\newline
\newglossaryentry{Bottleneck}{name={Bottleneck},description={Der Bottleneck ist, in der Softwareentwicklung, die Stelle, an der die Leistung der gesamten Anwendung am meisten eingeschränkt wird.}}
Wie oft, kann die Anwendung theoretisch und praktisch pro Sekunde aktualisiert werden?
Für jede Kombination aus Azure App und E-Mail-Adresse, dürfen 10000 Anfragen pro 10 Minuten gemacht werden.
Dies sind 16,6 Anfragen pro Sekunde.
Praktisch wurden eine bis drei Anfragen, je nach Bedarf, pro drei Sekunden gemacht.
Pro Sekunde wären das also 0,33 bis 1 Anfragen.
Einerseits ist das schnell genug für die Anwendung und gibt den langsamen Geräten, genug Zeit, die Anwendung zu aktualisieren, andererseits hat man so genug Anfragen übrig, falls jemand sein Konto mit mehreren Geräten benutzt oder dies in Kombinationen mit anderen Applikationen verwendet.
Sollte dieses Limit überschritten werden, verlangsamt Microsoft die Anzahl an neuen Antworten auf die Anfragen.
Da Microsoft auch Zeit benötigt, um die Anfragen zu bearbeiten und die neuen Daten bei sich zu verarbeiten, würde der Anwender den Unterschied selten bemerken.
Der \gls{Bottleneck} der Anwendung ist also in diesem Fall die Anzahl an Anfragen, die Microsoft, in ihren Servern, verarbeiten kann oder will.
\newline
\newline
\subsection{Fazit}\label{subsec:fazit}
Die Microsoft Graph API ist eine sehr mächtige Schnittstelle, die es ermöglicht, mit einer Vielzahl an Microsoft Diensten zu interagieren.
Für die Anwendung wurde die Schnittstelle genutzt, um Termine zu erstellen, zu löschen und zu aktualisieren.
Außerdem wurde die Schnittstelle genutzt, um die Logos der anderen Termine zu erhalten.
Die Schnittstelle wurde nicht vollständig genutzt, da die Anwendung nicht alle Funktionen benötigt, aber bietet für die Zukunft viele Erweiterungsmöglichkeiten.
Zudem ist die API recht neu und wird ständig weiterentwickelt~\cite{microsoft-graph-api-version}.
Es Anwendung wurde mit der Version 1.0 der API entwickelt.
Für den Anwendungsfall des Kunden ist die API ausreichend, da die Anwendung nur Termine erstellen, löschen und aktualisieren muss.
Der einzige Nachteil ist, dass Azure AD, die Authentifizierung, für Ressourcenkonten nicht unterstützt, die beispielsweise, in Teams, genutzt werden.
Daher muss der Anwender sich mit einem Microsoft-Konto anmelden, welcher die Ressource repräsentieren soll.
Dies kann zu Verwirrung führen, da die Terminologie impliziert, man könne sich mit einem Ressourcenkonto anmelden.
\newline
\newline
Trotzdem ist die Anwendung eine gute Basis für weitere Erweiterungen und war die richtige Wahl für diesen Kundenauftrag.
Ressourcen- und Terminplanung wurde mit der Anwendung vereinfacht, für Office365 Nutzer, definitiv vereinfacht.
Ein Anwender braucht nur noch drei Klicks, um einen Termin zu erstellen und muss nicht mehr zwischen verschiedenen Kalendern hin und her wechseln.
Zudem benötigt er nur ein Mal auf den Bildschirm zu schauen, um zu sehen, wann die Ressource, die durch die Software repräsentiert wird, frei ist und wann der Anwender gegebenenfalls einen Termin besitzt (Anhand der Logos).
\newline
\newline
Die Anwendung kann auch für menschliche Ressourcen genutzt werden, da die Anwendung keine Unterscheidung zwischen menschlichen und nicht-menschlichen Ressourcen macht.
Solche Entscheidungen obliegen dem Anwender.
\newline
\newline
Zusammenfassend kann gesagt werden, dass die Anwendung eine gute Basis für weitere Erweiterungen ist und die Anforderungen des Kunden, so erfüllt.
Damit geht einher, dass die Ziele der Praxisarbeit erreicht wurden.
Zu guter letzt, folgt noch ein Bild des Gerätes mit der Anwendung, welches im Kundenbetrieb genutzt wird.
\newline
\par\vspace{1cm}
    \centering
    \includegraphics[width=0.5\textwidth]{Bilder/FertigesProdukt}
    \caption{Fertiges Produkt - Gerät mit der Anwendung}
    \label{fig:fertiges-produkt}
\newline
\newline
\raggedright
\newpage
\section{Literaturverzeichnis}\label{sec:literaturverzeichnis}