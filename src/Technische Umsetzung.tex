%! Author = mboehme
%! Date = 23.02.2023

\section{Technische Umsetzung}\label{sec:technische-umsetzung}
%Entwicklung fängt schon vorher an.
\subsection{Prototyp}\label{subsec:prototyp}
Basierend auf dem Mockup ~\ref{fig:Mockup} wurde ein Prototyp erstellt, welcher testen sollte, ob Grundfunktionalitäten wie die Terminbuchung und die Darstellung der Termine funktionieren.
Dies wurde mithilfe von vuejs aufgesetzt.
\newline
Der Prototyp funktionierte wie erwartet und konnte somit als Grundlage für die Weiterentwicklung dienen.
\subsection{Benutzte Hardware und Software}\label{subsec:benutzte-hardware-und-software}
\newline
Die Hardware ist ein Philips~\cite{10BDL4551T/00} Bildschirm.
Dieser Bildschirm ist ein 10 Zoll (ca. 25 cm diagonal) Touchscreen.
Es hat eine Auflösung von 1280x800 Pixeln und benutzt Android 7 als Betriebssystem.
\newglossaryentry{Digital Signage}{name={Digital Signage}, description={Digital Signage ist eine ferngesteuerte, digitale Anzeige. Sie wird meistens in öffentlichen Räumen, wie zum Beispiel Flughäfen, Einkaufszentren oder Bahnhöfen, verwendet.}}
Der Bildschirm ist für Digital Signage gedacht und kann so auch als solches verwendet werden.
\newglossaryentry{SICP}{name=SICP, description={SICP steht für Serial (Ethernet) Interface Communication Protocol.
SICP ist ein Protokoll, welches es ermöglicht, mit dem Bildschirm direkt zu kommunizieren, ohne mit dem Android Betriebssystem zu interagieren.}}
Die eingebauten RGB LED Strips können per~\gls{SICP} angesteuert werden.
Dafür wird Dory-node.js genutzt, um auf einen lokalen Server zu hören und die Anfragen an den Bildschirm weiterzuleiten.
Es wird dafür keine Internetverbindung benötigt.
Dokumentationen für SICP öffentlich schwierig zu finden.
Befehle werden als Hexadezimal Werte geschrieben und an den Bildschirm gesendet.
%Quellenangabe
Die Verschlüsselung passiert, indem ein XOR aller vorausgehenden Bytes genommen und als Prüfsumme mitgegeben wird.
Diese Verschlüsselung ist von Phillips definiert.
Sie soll lediglich verhindern, dass jemand versehentlich falsche Befehle an den Bildschirm sendet.
\newline
\newline
\newglossaryentry{IDE}{name=IDE, description={Eine IDE ist eine Entwicklungsumgebung, die es ermöglicht, Software zu entwickeln.}}
Zur Entwicklung der Anwendung, wird eine~\gls{IDE} benötigt.
Als JavaScript Framework wird Vue.js verwendet, da es sich um eine Single Page Application handelt und Vue.js darauf spezialisiert ist.
\newline
\newglossaryentry{Node.js Server}{name={Node.js Server}, description={Ein Node.js Server ist ein Server, der auf Node.js basiert und JavaScript Code ausführt.}}
Um eine Kommunikation zwischen der Anwendung und dem Bildschirm zu ermöglichen, wird ein~\gls{Node.js Server} benötigt.
Dieser wird per Dory-node.js lokal gehostet.
\newglossaryentry{NPM}{name=NPM, description={NPM steht für Node Package Manager.}}
\newline
Damit die Authentifizierung funktioniert und Prozesse, wie das Einloggen, vereinfacht werden können, wird microsoft-mgt~\cite{Microsoft-mgt-npm} als~\gls{NPM} Package verwendet.
Dies beinhaltet das Microsoft Graph Toolkit~\cite{Microsoft-Graph-Toolkit}, welches alle Microsoft Graph API Komponenten enthält.
\newline
Für eine hohe Kompatibilitätsgewährleistung, wird Babel~\cite{Babel} und Webpack~\cite{Webpack} verwendet.
Babel stellt sicher, dass der Code in allen Browsern lauffähig ist und Webpack sorgt dafür, dass die Anwendung schnell lädt.
\newline
Um eine ordentliche Versionierung zu betreiben, wird Git, welches ein dezentrales Versionskontrollsystem ist, genutzt.
Es ermöglicht, Änderungen an Dateien zu verfolgen und zu verwalten.
Dies wird in Zusammenhang mit GitLab verwendet, da dies der derzeitige, etablierte Standard der DooH media GmbH ist.
\newline
\newglossaryentry{Linter}{name=Linter, description={Ein Linter ist ein Programm, das es ermöglicht, Code auf Fehler zu prüfen.}}
Zum Vermeiden von schlechter Codequalität und syntaktischen Fehlern, die der Compiler nicht erkennen kann, wird EsLint~\cite{EsLint} verwendet.
EsLint ist ein~\gls{Linter}, der es ermöglicht, JavaScript Code zu analysieren und zu formatieren.
Da Testfälle bei JavaScript Projekten nicht vor der Ausführung der Anwendung möglich sind, ist es wichtig, dass der Code einheitlich ist und keine syntaktischen Fehler enthält.
Deshalb achtet EsLint auf die Einhaltung von Regeln, damit erst gar keine Fehler entstehen, die durch falsche Syntax entstehen.
Beispielsweise wird überprüft, ob Variablen, welche deklariert werden, auch verwendet werden.
\newline
\newline
\subsection{Grobe Planung}\label{subsec:grobe-planung}
\newglossaryentry{deployen}{name=deployen,description={Eine Anwendung zu deployen bedeutet, sie auf einem Server zu installieren und zu konfigurieren, sodass sie für den Benutzer verfügbar ist.}}
Die Software wird als eine Azure App deployt (~\gls{deployen}).
\newglossaryentry{localhost}{name=localhost,description={Die lokale Adresse des Computers, auf dem die Anwendung läuft.}}
\newglossaryentry{cookie cache}{name=cookie cache,description={Ein Cookie ist eine kleine Textdatei, die von einem Webserver auf dem Computer des Benutzers gespeichert wird. Cookies werden verwendet, um Informationen über den Benutzer zu speichern.}}
Diese lässt ~\gls{localhost} Verbindungen zu und gibt einem die Möglichkeit das ganze als Single Page Application zu entwickeln, mit einem lokalen ~\gls{cookie cache} für den eingeloggten Account.
Deshalb agiert die Anwendung erstmal nur als ein Gerüst, welches darauf wartet, dass der Benutzer sich einloggt.
Sobald sich der Benutzer einloggt, wird die Anwendung mit den Daten des Benutzers gefüllt.
Somit ist keine Individualisierung der Seite für verschiedene Nutzer notwendig.
\newline
\newglossaryentry{Azure AD}{name=Azure AD,description={Azure Active Directory ist ein Verzeichnisdienst, der von Microsoft entwickelt wurde.}}
Der wichtigste Faktor für die Entscheidung, die Microsoft Graph API zu nutzen war, dass mithilfe von ~\gls{Azure AD}, die Authentifizierung und Autorisierung von Benutzern erlaubt.
\newglossaryentry{Dory-node.js}{name=Dory-node.js,description={Dory-node.js ist eine Android App, die es ermöglicht, node.js Server auf einem Android Gerät zu hosten.}}
Es wird, lokal, mithilfe von~\gls{Dory-node.js} gehostet.
\newline
%Das Wort "Player" einführen
\newglossaryentry{Player}{name=Player,description={Ein Player ist ein Digital Signage Gerät, welches Inhalte anzeigt.}}
Der ~\gls{Player} benötigt auf der OMS den Content-Typen \("\)Web\("\),
\newline
mit der URL: \("\)http://localhost:3000/content\("\).
Diese Seite wird dann lokal vom Player aufgerufen, worauf der Dory-node.js Server antwortet und die tatsächliche Seite anbietet, die lokal auf dem Bildschirm liegt.
%Bitte im Soll-Zustand einführen was oAuth ist
So ist die URL durchgehend, auf allen Geräten identisch, damit die OAuth 2.0 URI Bedingungen erfüllt sind.
Ohne diese Bedingungen, wäre es nicht möglich, sich bei der Anwendung anzumelden.
\newline
\newline
\newline


\subsection{Iterative Entwicklung der groben Planung}\label{subsec:iterative-entwicklung-der-groben-planung}
Es wurde eine Azure App erstellt, die die Authentifizierung der Anwendung und des Benutzers übernimmt.
\newline

Dort wird der Benutzer eingeloggt und die Anwendung leitet ihn auf die Seite weiter, die er vorher besucht hat, welche in diesem Fall, die Raumbuchungsseite ist.
\newline
Es wurde jede Woche Rücksprache mit dem Kunden gehalten, um die Anforderungen zu besprechen und zu erfüllen.
Aber auch intern wurde Rücksprache gehalten, was denn sinnvoll ist und was nicht.
\newline
%Iterative Entwicklung mit Kundenentwicklung. Vllt eigenes Kapitel
So wurde jede Woche die Anwendung weiterentwickelt und verbessert.
Teilweise wurden auch neue Features hinzugefügt, die nicht initial erwünscht, aber sinnvoll waren.
Manche Features wurden auch wieder entfernt, da sie nicht sinnvoll waren oder von anderen Features abgedeckt wurden.
\newline
%Ablauf der Entwicklung
\newline
\subsection{Rest-Anfragen}\label{subsec:rest-anfragen}

\newline
Um sicherzustellen, dass die Anfragen ISO 8601 konform sind, wird die Zeitzone bei jeglichen Anfragen mit angegeben.
Um eine Terminbuchung vorzunehmen, wird per REST API eine Anfrage an die Microsoft Graph API gesendet.
\newline
Diese sieht wie folgt aus:
\newline
\begin{lstlisting}[language=JavaScript,label={lst:JavaScript REST API Anfrage}]
"https://graph.microsoft.com/v1.0/me/calendar/events"
\end{lstlisting}
\newglossaryentry{POST}{name=POST,description={POST ist eine HTTP Methode, die Daten an einen Server sendet.}}
\newglossaryentry{Authorization}{name=Authorization,description={Authorization ist ein HTTP Header, der einen Schlüssel enthält, um die Anfrage zu authentifizieren.}}
\newglossaryentry{Content-Type}{name=Content-Type,description={Content-Type ist ein HTTP Header, der angibt, welche Art von Daten in der Anfrage enthalten sind.}}
\newglossaryentry{Header}{name=Header,description={Ein Header ist ein Teil einer HTTP Anfrage, der zusätzliche Informationen enthält.}}
Diese wird per POST Methode an die Microsoft Graph API gesendet.
Mehrere Header sind generell bei allen Fragen notwendig.
Einerseits ein Authorization Header, der den Access Token enthält, der für die Authentifizierung des Benutzers notwendig ist, und andererseits ein Content-Type Header, der auf application/json gesetzt ist.
\newline
Im Körper der Anfrage werden die Daten im JSON Format mitgegeben.
Diese enthalten Daten wie den Titel, die Startzeit, die Endzeit, die Beschreibung und die Zeitzone.
\newpage
\begin{lstlisting}[language=JavaScript,label={lst:JavaScript REST API Körper}]
{
    "subject": "Meeting",
    "start": {
        "dateTime": "2019-06-06T10:00:00",
        "timeZone": "Europe/Berlin"
    },
    "end": {
        "dateTime": "2019-06-06T11:00:00",
        "timeZone": "Europe/Berlin"
    },
}
    \end{lstlisting}
\newline
\newline
Um herauszufinden, welche Termine frei sind und welche nicht, wird dafür ebenso eine Anfrage an die Microsoft Graph API gesendet.
Diese benutzt eine andere URL und andere Daten im Körper (Siehe Kapitel~\ref{subsubsec:abfrageoptimierung}).
\subsubsection{Fehlerbehandlung}\label{subsubsec:fehlerbehandlung-REST}
Falls mehrere Anfragen, welche Informationen aktualisieren sollen, nacheinander fehlschlagen, wird eine Fehlermeldung angezeigt, die den User darauf hinweist, dass die Verbindung zum Server derzeit nicht gewährleistet werden kann.
Diese wird groß und prominent angezeigt, damit es nicht übersehen werden kann und deutlich wird, dass jegliche Daten die derzeit angezeigt werden, nicht aktuell sind.
Dementsprechend wird die Fehlermeldung entfernt, sobald eine Anfrage erfolgreich war.
\newline
\newline
%Probleme gehören eigentlich woanders hin
\subsection{Timer}\label{subsec:timer}
Der Timer, welcher die übrig bleibende Zeit bis zum Ende des jetzigen Termins anzeigt funktioniert, wie folgt:
\newline
\newline
Es wird die Zeit bis zum Ende den jetzigen Termin berechnet, ausgehend von der Zeit, die zum Anfang des jetzigen Termins vergangen ist, und in Millisekunden umgerechnet.
Dann wird eine sich drehende Animation erstellt, die für diese Dauer abläuft.
\newline
Da die Dauer angepasst werden kann, muss bei Datenänderungen geprüft werden, ob sich die Dauer verändert hat.
%Glossar local storage
\newglossaryentry{Local Storage}{name=Local Storage,description={Local Storage ist ein Speicherort, in Browsern, welcher Daten lokal speichert.}}
Die Daten dafür werden im~\gls{Local Storage} gespeichert.
Falls eine Änderung stattgefunden hat, wird die Animation neu berechnet.
Bei der Änderung werden, damit die Animation visuell nicht von vorne anfängt, die Animationsdauer und die vergangene Zeit als neue Animationsdauer aufaddiert und die vergangene Zeit als negative Animationsverzögerung gesetzt, damit die Animation visuell betrachtet dort ist, wo sie relativ zur neuen Dauer sein sollte.
\newline
\newline
Die Formel lautet, wie folgt:
\newline
\newline
\begin{equation}
\begin{aligned}
    \text{Vergangene Zeit} &= \text{Jetzt} - \text{Start} \\
\text{Animationsdauer} &= \text{Ende} - \text{Start} \\
    \text {Animationsverzögerung} &= \text{-Vergangene Zeit} \\
\end{aligned}\label{eq:equation}
\end{equation}
\newline
\newline
Die Berechnung wurde wie folgt implementiert:
\newline
\newline
\newpage
\begin{lstlisting}[language=JavaScript,label={lst:JavaScript Timer}]
let currentEventBeginningTime = localStorage.getItem('currentEventBeginningTime');
let timePassed = (new Date() - new Date(currentEventBeginningTime + 'Z')) / 1000;
let animationDuration = ((new Date(currentEventEndTime + 'Z') - new Date(currentEventBeginningTime + 'Z')) / 1000);
firstHandSpan.style.animationDuration = animationDuration  + 's';
secondHandSpan.style.animationDuration = animationDuration + 's';
firstHandSpan.style.animationDelay = -timePassed + 's';
secondHandSpan.style.animationDelay = -timePassed  + 's';
\end{lstlisting}
\newline
\newline
Um die Animation während ihrer Laufzeit zu ändern, wird ein sogenannter Reflow ~\footcite{JavaScriptAnimationsReflow} erzwungen, indem die offsetWidth Eigenschaft abgefragt, die Animationsdauer und -verzögerung neu gesetzt und der Animationsname erst entfernt und dann wieder hinzugefügt wird.
\newline
\newline
\subsection{Persistente Datenspeicherung}\label{subsec:persistente-datenspeicherung}
Um die Daten der Anwendung zu speichern, wurde für kleinere Datensätze der Local Storage verwendet.
Dieser ist in jedem Browser verfügbar und darf bis zu 5 MB groß sein.
\newline
\newline
Für größere Datensätze wurde IndexedDB~\cite{IndexedDB} verwendet.
\newglossaryentry{caniuse}{name={caniuse.com},description={Caniuse is a website that shows you browser support for various features and includes references to the relevant specifications.}}
Dies ist eine NoSQL Datenbank, die in jedem gängigen Browser~\cite{caniuse-indexedDB} (\gls{caniuse}) verfügbar ist.
Die maximale Größe bei Chrome beträgt 80\(\%\) des verfügbaren Speichers.
Da es im Internet viele, sich widersprechende Angaben zu dieser Größe gibt, wurde diese manuell, mit folgendem Code, in der Konsole des Browsers, getestet:
\newline
\newline
\begin{lstlisting}[language=JavaScript,label={lst:JavaScript IndexedDB Speichergröße}]
    if (navigator.storage && navigator.storage.estimate) {
    const quota = await navigator.storage.estimate();
    // quota.usage -> Number of bytes used.
    // quota.quota -> Maximum number of bytes available.
    const percentageUsed = (quota.usage / quota.quota) * 100;
    console.log("You've used ${percentageUsed}% of the available storage.");
    const remaining = quota.quota - quota.usage;
    console.log("You can write up to ${remaining} more bytes.");
    }
\end{lstlisting}
\newline
\newline
Die Festplatte, auf der dies getestet wurde, hatte ca. 828 GB freien Speicher.
Das Ergebnis lautete, wie folgt:
\newline
\newline
\begin{lstlisting}[language=JavaScript,label={lst:JavaScript IndexedDB Speichergröße Ergebnis}]
    You've used 0.000002393592594674544% of the available storage.
    You can write up to 599726105785 more bytes.
\end{lstlisting}
\newline
\newline
%equation
\begin{equation}
    \begin{aligned}
       \text{verbleibender Speicher} &= \text{599726105785 Bytes}\\
       \newline
       \text{599726105785 Bytes} \text{ * 10}\textsuperscript{-9} &= \text{599,726105785 Gigabytes}\\
        \newline
        \newline
    \end{aligned}
\end{equation}
\newline
\newline
Das sind etwas über 70 \%\) der verfügbaren Speicherkapazität.
Hier ist davon auszugehen, dass die restlichen 10 \%\) für andere Daten schon verwendet werden.
Zudem ist zu betrachten, dass die \"\).\"\) Schreibweise für Dezimalzahlen in JavaScript verwendet wurde, während wiederum die \"\),\"\) Schreibweise für die mathematische Gleichung verwendet wurde.
Dies hat den Hintergrund, dass die \"\),\"\) Schreibweise in JavaScript nicht grundsätzlich unterstützt wird, beziehungsweise, die Konvention beim Programmieren verlangt, dass man auf Englisch programmiert und im Englischen die Dezimalzahlen mit einem Punkt geschrieben werden.
\newline
\newline
Die Datenbank befindet sich aufgrund ihrer geringen Komplexität in der 5. Normalform.
Der Schlüssel besteht aus dem Namen der Firma, zu der, das Logo gehört und gibt das nicht Primärattribut zurück, welches das Bild, in Base64, enthält.
Schlüssel müssen eindeutig sein.
IndexedDB unterstützt die Eigenschaft \"\)unique\"\), welche dafür sorgt, dass der Schlüssel eindeutig ist.
Jedoch gehen wir auch manuell noch einmal sicher, dass der Schlüssel eindeutig ist, also dass sich dieser, noch nicht in der Datenbank befindet.
Eine weitere Aufteilung würde zu Informationsverlust führen, da der Firmenname eindeutig ist.
\newline
\newline
Zwar gibt es die Möglichkeit die Bilder für Firmen, zu updaten, welcher der User auch so wahrnimmt, jedoch wird der Schlüssel des alten Bildes entfernt und ein neuer Schlüssel mit dem neuen Bild hinzugefügt.
Dies ist notwendig, da der Schlüssel eindeutig sein muss.
Somit löschen wir den alten Eintrag und fügen einen neuen hinzu.
Unabhängig von der Interaktion mit der Datenbank, wird sichergestellt, dass sinnfreie Anfragen an die Datenbank nicht gesendet werden.
Falls der User beispielsweise ein Bild hochlädt, welches bereits in der Datenbank, für genau den gleichen Schlüssel, vorhanden ist, wird die Anfrage an die Datenbank nicht gesendet oder falls der User versucht ein Bild aus der Datenbank zu entfernen, welches gar nicht existiert.
\newline
\newline
%Tabelle erläutern
\subsection{Azure Authentifizierung}\label{subsec:Azure Authentifizierung}
Um die Authentifizierung zu realisieren, wurden folgende Berechtigungen benötigt:
\newline
\newline
%    \begin{table}
    \centering
%table should not be wider than 0.8\textwidth
\small
    \begin{tabularx}{\textwidth}{|X|X|}
        \toprule
        \textbf{Calendars.Read} & \textbf{Lesezugriff auf Benutzerkalender}\\
        \hline
        Calendars.Read.Shared  & Benutzer und freigegebene Kalender lesen\\
        \hline
        Calendars.ReadWrite  & Verfügt über Vollzugriff auf Benutzerkalender.\\
        \hline
        Calendars.ReadWrite.Shared  & Benutzerdefinierte und freigegebene Kalender lesen und schreiben\\
        \hline
        email  & E-Mail-Adresse von Benutzern anzeigen \\
        \hline
        IMAP.AccessAsUser.All  & Read and write access to mailboxes via IMAP.\\
        \hline
        Mail.Read  & Lesezugriff auf Benutzer-E-Mails\\
        \hline
        Mail.Send  & E-Mails unter einem anderen Benutzernamen senden\\
        \hline
        Mail.Send.Shared  & E-Mails im Namen anderer Benutzer senden\\
        \hline
        offline\_access  & Zugriff auf Daten beibehalten, für die Sie Zugriff erteilt haben\\
        \hline
        OnlineMeetings.Read  & Read user's online meetings\\
        \hline
        OnlineMeetings.ReadWrite  & Read and create user's online meetings\\
        \hline
        openid  & Benutzer anmelden\\
        \hline
        profile  & Grundlegendes Profil von Benutzern anzeigen\\
        \hline
        User.Read  & Anmelden und Benutzerprofil lesen\\
        \bottomrule
    \end{tabularx}
    \caption{Berechtigungen für die Authentifizierung}\label{tab:Berechtigungen für die Authentifizierung}
%    \end{table}
\newline
\newline
\raggedright
\normalsize
Diese werden beim erstmaligen Login, vom User, seitens Microsoft, angefordert.
Ohne die Zustimmung des Users, wird die Anwendung nicht entsprechend den Anforderungen funktionieren und weiterhin darauf hinweisen, dass der User sich anmelden soll.
Alle Berechtigungen werden als delegierte Berechtigungen angefordert.
Das bedeutet, dass die Anwendung, für den Benutzer, die Handlung durchführen darf, die durch die Berechtigung angefordert wurde.
\newglossaryentry{IMAP}{name=IMAP,description={Internet Message Access Protocol. Ein Protokoll, welches es ermöglicht, E-Mails von einem Mailserver abzurufen.}}
Mit diesen Berechtigungen kann sichergestellt werden, dass auch Exchange-Konten, welche eine eigene~\gls{IMAP}-Adresse haben, alle Funktionalitäten der Anwendung nutzen können.
%\section{Verifikation}\label{sec:Verifikation}
%Damit verifiziert werden kann, dass die Anwendung so funktioniert, wie sie soll, wurden einige Tests durchgeführt.
\subsection{Performanztests}\label{subsec:Pefromanztests}
%Tests brauchen Namen
Die Performance einer \gls{SPA} zu messen ist trotz der geringen Komplexität der Anwendung, nicht simpel.
Auch die gängige Total-Blocking-Time Messung ist nicht zwingend sinnvoll, da die Anwendung nicht für den Nutzer, während der Nutzung, blockiert ist, sondern lediglich die Daten vom Server geladen werden, um anschließend die Nutzung der Anwendung sinnvoll zu ermöglichen.
\newglossaryentry{RAIL Modell}{name={RAIL Modell},description={Das RAIL Modell ist ein Modell, welches die Performance von Webseiten beschreibt. Es beschreibt, dass eine Webseite in 5 Bereiche unterteilt werden kann, welche die Performance beeinflussen. Diese sind: Antwort, Animation, Leerlauf, Laden und Bilder pro Sekunde.}}
Daher wurden manuell einige Tests durchgeführt, um die Performance zu messen und in Zusammenhang mit dem RAIL Modell~\cite{RAILModell} bewertet.
Es sollte dafür berücksichtigt werden, dass der Prozessor des Tablets, welches die Anwendung verwendet, ein 32 Bit Prozessor ist, welcher im Jahr 2013 auf den Markt kam.
\newline
\newline
\subsubsection{Initiales Laden}\label{subsubsec:initiales-laden}
\newline
\newline
Die~\gls{TBT} der Applikation liegt, aufgrund von hoher Komplexität, bei ca. 500ms.
Somit ist das Laden der Webseite akzeptabel~\cite{TotalBlockingTime}, da die Total-Blocking-Time nicht die Grenze von 600ms überschreitet.
Das RAIL Modell erlaubt sogar Ladezeiten von bis zu 5 Sekunden beim initialen Laden der Webseite, falls innerhalb dieser 5 Sekunden die Interaktivität mit der Webseite gegeben ist.
Die Hardware, die hier verwendet wurde, besitzt einen 32 Bit Prozessor, welcher im Jahr 2013 auf den Markt kam.
\newline
\subsubsection{Interaktivität/Leerlauf}\label{subsubsec:test-1}
\newline
\newline
Wie lange braucht die Anwendung, um nach einem Klick auf den Button \("\)+\("\) das Terminerstellungs-Menü zu öffnen?
\gls{Event-Listener}{name={Event-Listener},description={Ein Event-Listener ist ein Objekt, welches auf bestimmte Ereignisse wartet und dann eine Funktion ausführt. Dieses Objekt braucht ein HTML Element, welches es überwachen soll.}}
Die Antwort auf diese Fragestellung wurde gemessen, indem der Event-Listener für den Button \("\)+\("\) registriert ausgab, wann die Taste betätigt und damit Zeit gemessen wurde, die vergeht, bis der \gls{Event-Listener} das Menü darstellt.
Es dauert im Durchschnitt 2ms, bis das Menü angezeigt wird.
Damit sind wir weit unter der 100ms Grenze, welche das RAIL Modell für Interaktivität mindestens empfiehlt und weit unter den 50ms, welche das RAIL Modell anstrebt.
\newglossaryentry{Leerlauf}{name={Leerlauf},description={Der Leerlauf ist der Zeitraum, in dem die Anwendung keine Aktionen des Nutzers begegnet. Diese Zeit kann für die Anwendung genutzt werden, um Daten zu laden oder Berechnungen durchzuführen.}}
Dies bedeutet also, dass der Leerlauf der Anwendung sich nicht beeinträchtigend auf die Echtzeit-Wahrnehmung des Nutzers auswirkt.
\newline
\newline
\subsubsection{Animationen}\label{subsubsec:test-2}
\newline
\newline
Wie viele Bilder pro Sekunde zeigt die Anwendung durchschnittlich an?
\newglossaryentry{LightHouse}{name={LightHouse},description={LightHouse ist ein Tool, welches von Google entwickelt wurde, um die Performance einer Website zu messen.}}
Dies wurde mithilfe von LightHouse über 60 Sekunden hinweg gemessen.
\newglossaryentry{Worst-Case}{name={Worst-Case},description={Der Worst-Case ist der Fall, in dem die Leistung der Anwendung am schlechtesten ist.}}
\newline
Um ein~\gls{Worst-Case} Szenario zu simulieren, wurden 30 Termine erstellt, welche unterschiedliche Gastlogos, inklusive GIFs, enthielten.
Ein aktueller Termin wurde initialisiert, um die Animation des Timers zu starten.
Hinzufügend wurde die Anwendung benutzt, um Termine zu buchen und zwischen dem dunklen und hellen Design zu wechseln.
Die Anwendung zeigt durchschnittlich 60 Bilder pro Sekunde an.
Da die maximale Wiederholrate des Bildschirms und des Browsers bei 60Hz liegen, besteht nicht die Möglichkeit, für diesen Anwendungsfall zu testen, ob noch mehr Bilder pro Sekunde angezeigt werden können.
Nichtsdestotrotz sind damit die Anforderungen des RAIL Modells für Bilder pro Sekunde erfüllt.
\newline
\newline
\subsubsection{Bottleneck}\label{subsubsec:test-bottleneck}
\newline
\newline
\newglossaryentry{Bottleneck}{name={Bottleneck},description={Der Bottleneck ist, in der Softwareentwicklung, die Stelle, an der die Leistung der gesamten Anwendung am meisten eingeschränkt wird.}}
Wir wollen betrachten wie oft die Anwendung theoretisch und praktisch pro Sekunde aktualisiert werden kann.
\newglossaryentry{Azure APP ID}{name={Azure APP ID},description={Azure APP ID ist eine eindeutige Kennung, welche von Microsoft vergeben wird, um eine Applikation zu identifizieren.}}
Für jede Kombination aus~\gls{Azure APP ID} und E-Mail-Adresse, dürfen 10000 Anfragen pro 10 Minuten gemacht werden.
Dies sind 16,6 Anfragen pro Sekunde.
Praktisch wurden eine bis drei Anfragen, je nach Bedarf, pro drei Sekunden durchgeführt.
Dieser Bedarf hängt davon ab, ob sich an den Daten, die bei Microsoft gespeichert sind, etwas geändert hat und daran, ob Bilder-Anhänge für Gäste vorhanden sind, die nochmal separat per REST API abgefragt werden müssen.
Pro Sekunde wären das also 0,33 bis 1 Anfragen.
Einerseits ist das schnell genug für die Anwendung und gibt den langsamen Geräten, genug Zeit, die Anwendung zu aktualisieren, andererseits hat man so genug Anfragen übrig, falls jemand sein Konto mit mehreren Geräten benutzt oder dies in Kombinationen mit anderen Applikationen verwendet.
Sollte dieses Limit überschritten werden, verlangsamt Microsoft die Anzahl an neuen Antworten auf die Anfragen.
Da Microsoft auch Zeit benötigt, um die Anfragen zu bearbeiten und die neuen Daten bei sich zu verarbeiten, würde der Anwender den Unterschied selten bemerken.
Der \gls{Bottleneck} der Anwendung ist also in diesem Fall die Anzahl an Anfragen, die Microsoft, in ihren Servern, verarbeiten kann.
\newline
\newline
\subsection{Funktionelle Tests}\label{subsec:funktionelle-tests}
\subsubsection{API Anfragen}\label{subsubsec:probleme-der-api}
Es wurden im Voraus Befehle im Microsoft Graph Explorer~\cite{Microsoft-Graph-Explorer} ausgeführt, um zu testen, ob die Anfrage funktioniert und ob die Daten in der gewünschten Form zurückgegeben werden.
Jedoch hat sich herausgestellt, dass der Microsoft Graph Explorer nicht korrekt funktioniert und teilweise Befehle nicht entgegennimmt, die laut Microsoft Dokumentation funktionieren sollten.
Evident ist dies, da die Anwendung korrekt funktioniert und die REST API Anfragen korrekt funktionieren und auch im Request-Response Zyklus keine Fehler auftreten.
\newline
\newline
Zudem war es initial nicht möglich, alle verfügbaren Termine, für den angegebenen Zeitraum, abzufragen.
Dies wurde gelöst, indem die maximale Anzahl an Teilnehmern, für einen Termin, welche im Körper der Anfrage übergeben werden, auf 99 gesetzt wurde.
Diese Lösung wurde durch einen Nutzer, in einem Forum, entdeckt.
Zwar ist dies nicht von Microsoft so vorgesehen, aber die Lösung beeinträchtigt die Anwendung nicht, da es ohnehin nicht mehr möglich ist, Teilnehmer hinzuzufügen.
\newline
\newline
%\subsubsection{Objektorientiertes Design}
%Hier sieht man ein Sequenzdiagramm, welches die Kommunikation zwischen dem Anwender und dem Server darstellt:
%\newline
%\newline
%\par\vspace{0.5cm}
%\begin{figure}[h]
%\centering
%\includegraphics[width=\textwidth]{Bilder/Objektorientiertes Design/Sequence diagram for ressource booking (3)}
%\par\vspace{0.5cm}\label{fig:figure}
%\end{figure}
%\justifying
%\newline
%\newline
