%! Author = mboehme
%! Date = 23.02.2023

\subsection{Technische Umsetzung}
\subsubsection{Grobe Planung}
Das Ganze wird als eine Azure App deployt.
Diese lässt localhost Verbindungen zu und gibt einem die Möglichkeit das ganze als Single Page Application zu entwickeln, mit einem lokalen cookie cache für den eingeloggten Account.
Somit ist keine Individualisierung notwendig.
\newline
Es wird, lokal, mithilfe von Dory-node.js gehostet.Das ganze wird als eine Azure App deployt.
Diese lässt localhost Verbindungen zu und gibt einem die Möglichkeit die Web-Applikation wird als Single-Page-Application, mit einem lokalen cookie cache für den eingeloggten Account, entwickelt.
Somit ist keine Individualisierung notwendig.
Zudem kann dann mit dem Phillips Display die Web-Applikation aufgerufen werden und als eine Art Model eines Model-View-Controllers verwendet werden.
\newline
Der Player braucht auf der OMS den Content-Typen \("\)Web\("\) mit der URL: \("\)http://localhost:3000/content\("\).
Diese Seite wird dann lokal vom Player aufgerufen, worauf der Dory-nodejs Server antwortet und die tatsächliche Seite anbietet, die lokal auf dem Display liegt.
So ist die URL immer die Gleiche, auf allen Geräten, damit Microsoft's redirect URI Bedingungen erfüllt werden können und Webserver kosten eingespart werden können, da Updates eher selten passieren sollten.
\newline
Zudem ist es auch für die Sicherheit der Nutzer so viel besser, da wir deshalb keinen Zugriff auf ihre Daten haben.
Zeiten müssen ISO 8601 Konform sein bei jeglichen API Anfragen an die Microsoft Graph API\@.
\newline
\pagebreak
\subsubsection{Benutzte Hardware und Software}
Hardware:
\begin{itemize}
    \item Philips 10BDL4551T/00
\end{itemize}
\newline
Die Hardware ist ein Philips 10BDL4551T/00 Display.
Dieses Display ist ein 10 Zoll (ca. 25 cm) Touchscreen.
Es hat eine Auflösung von 1280x800 Pixeln und benutzt Android.
Dieses Display ist für Digital Signage gedacht und kann so auch als solches verwendet werden.
Die eingebauten RGB LED Strips können per SICP angesteuert werden.
\newline
\newline
Software:
\begin{itemize}
    \item Vue.js 3.x
    \item Microsoft Graph API per NPM Package
    \item Dory-Node.js
    \item Azure App
    \item Babel
    \item Webpack
    \item Node.js
    \item NPM
    \item Git
    \item EsLint
    \item IntelliJ IDEA
\end{itemize}
\newline
Vue.js ist ein JavaScript Framework und transpiliert die Vue.js Dateien in JavaScript, damit sie von jedem Browser ausgeführt werden können.
Es ist darauf spezialisiert, Single Page Applikationen zu entwickeln.
\newline
\newline
Dory-Node.js ist ein Node.js Server, der es ermöglicht, eine Web-Applikation lokal zu hosten.
Dies wird benötigt, um einerseits die Raumbuchungsseite lokal anzubieten, damit die absoluten Redirect-URI Bedingungen von Microsoft und oAuth 2.0 erfüllt werden können und andererseits, um die Anwendung auf dem Display zu hosten und als Schnittstelle zwischen der Seite und dem den LED-Strips des Displays zu agieren.
\newline
\newline
Die Microsoft Graph API wird per NPM Package verwendet, um die Anfragen an die Microsoft Graph API zu vereinfachen und den User einzuloggen.
\newline
\newline
Babel ist ein JavaScript Compiler, der es ermöglicht, moderne JavaScript Features zu verwenden, die von älteren Browsern nicht unterstützt werden, um so die Kompatibilität zu erhöhen.
\newline
\newline
Webpack ist ein JavaScript Bundler, der es ermöglicht, mehrere JavaScript Dateien zu einer einzigen zusammenzufassen, um so die Ladezeiten zu verkürzen.
\newline
\newline
Node.js ist ein JavaScript Runtime, der es ermöglicht, JavaScript Code auszuführen, ohne einen Browser zu benötigen.
Es ist das sogenannte Backend dieser Anwendung.
Dieser wird per Dory-Node.js gehostet.
\newline
\newline
NPM ist ein Package Manager, der es ermöglicht, JavaScript Packages zu installieren und zu verwalten.
\newline
\newline
Git ist ein Versionskontrollsystem, das es ermöglicht, Änderungen an Dateien zu verfolgen und zu verwalten.
\newline
\newline
EsLint ist ein Linter, der es ermöglicht, JavaScript Code zu analysieren und zu formatieren.
Da Testfälle bei JavaScript Projekten nicht immer vollständig möglich sind, ist es wichtig, dass der Code einheitlich ist und keine Fehler enthält.
Deshalb achtet EsLint auf die Einhaltung von Regeln, damit erst gar keine Fehler entstehen, die der JavaScript Compiler nicht erkennen kann.
Beispielsweise wird überprüft, ob Variablen deklariert wurden, bevor sie verwendet werden und anders herum.
\newline
\newline
IntelliJ IDEA ist eine IDE, die es ermöglicht, JavaScript Code zu schreiben und zu debuggen.
\newline
\newline

\subsubsection{Weiterentwicklung der groben Planung}
Es wurde eine Azure App erstellt, die die Authentifizierung der Anwendung und des Users übernimmt.
\newline
Dort wird der User eingeloggt und die Anwendung leitet ihn auf die Seite weiter, die er vorher besucht hat, welche in diesem Fall, die Raumbuchungsseite ist.
\newline
Es wurde jede Woche Rücksprache mit dem Kunden gehalten, um die Anforderungen zu besprechen und zu erfüllen.
Aber auch intern wurde Rücksprache gehalten, was denn sinnvoll ist und was nicht.
\newline
So wurde jede Woche die Anwendung weiterentwickelt und verbessert.
Teilweise wurden auch neue Features hinzugefügt, die nicht im Pflichtenheft standen, aber sinnvoll waren.
Manche Features wurden auch wieder entfernt, da sie nicht sinnvoll waren oder von anderen Features abgedeckt wurden.
\newline
%Ablauf der Entwicklung
\newline
\subsubsection{Ablauf der Entwicklung}
Nachdem der erste Prototyp fertig war und Rücksprache gehalten wurde, wurde angefangen die Anwendung zu entwickeln.
Design und Funktionalität wurden dabei partiell parallel entwickelt, wobei die Funktionalität immer Priorität hatte.
Über 95 Commits wurden die Änderungen an der Anwendung festgehalten.
\newline
Es wurde ein Testgerät benötigt, um die Anwendung zu testen.
Ein Phillips 10BDL4551T/00 Display wurde dafür an die Wand gehängt und mit dem Internet verbunden.
Zudem wurde es so eingerichtet, wie es auch beim Kunden laufen soll.
\newline
Es wurde täglich aktiv genutzt, um so Fehler zu finden und Feedback zu geben.
Das Feedback wurde dann, falls sinnvoll, in die Anwendung eingearbeitet.
Solche praktischen Tests sind sehr wichtig, um Intuitivität und Benutzerfreundlichkeit zu gewährleisten.
Es war einerseits hilfreich, um vorgesehene Abläufe zu testen, aber auch um Fehler zu finden, die nicht vorgesehen waren, indem man monkey-testing betreibt.
\newline
Mithilfe der oData v4 API wurden die Daten aus der jeweiligen Microsoft Datenbank im Voraus gefiltert, um so nicht notwendige Daten zu vermeiden und Performance drastisch zu erhöhen, als auch Datenvolumen zu sparen.
Die Daten wurden dabei in JSON Format zurückgegeben.
Die oData v4 API verhält sich dabei wie eine SQL Datenbank, wobei die Daten in Tabellen gespeichert sind.
Es wurden die Klauseln \("\$\)filter\("\) und \("\$\)top\("\) verwendet, um nur Termine für den aktuellen Tag und nächsten Tag zu erhalten und dies einzuschränken auf die top 300 Termine, falls jemand versucht das System zu überlasten.
Eine Sortierung ist nicht notwendig, da die Termine bereits nach Startzeit sortiert sind.
Hier sieht man die oData v4 API Anfrage:
\newline

%add javascript code formatting
\begin{lstlisting}[language=javascript,label={lst:JavaScript oData v4 API Anfrage}]
     let url =  "https://graph.microsoft.com/v1.0/me/findMeetingTimes/?$filter=start/dateTime" +  "ge"  + "${todayDate} and end/dateTime le ${tomorrowDate}&$top=300";
\end{lstlisting}
\newline
\newline
\subsubsection{Timer}
Der Timer, welcher die übrig bleibende Zeit bis zum nächsten Termin anzeigt funktioniert, wie folgt:
\newline
\newline
Es wird die Zeit bis zum Ende den jetzigen Termin berechnet, ausgehend von der Zeit, die zum Anfang des jetzigen Termins vergangen ist, und in Millisekunden umgerechnet.
Dann wird eine sich drehende Animation erstellt, die diese für diese Dauer abläuft.
\newline
Da die Dauer angepasst werden kann, muss bei Datenänderungen die Animation neu berechnet werden.
Damit dies nicht ohne Grund passiert, wird geprüft, ob die Dauer sich geändert hat.
Die Daten dafür werden im Local Storage gespeichert.
Falls eine Änderung stattgefunden hat, wird die Animation neu berechnet.
Bei der Änderung werden, damit die Animation visuell nicht von vorne anfängt, die Animationsdauer und die vergangene Zeit als neue Animationsdauer aufaddiert und die vergangene zeit als negative Animationsverzögerung gesetzt, damit die Animation dort ist, wo sie relativ zur neuen Dauer sein sollte.
\newline
\newline
Die Formel lautet, wie folgt:
\newline
\newline
\begin{equation}
\begin{aligned}
    \text{Vergangene Zeit} &= \text{Jetzt} - \text{Start} \\
\text{Animationsdauer} &= \text{Ende} - \text{Start} \\
    \text {Animationsverzögerung} &= \text{-Vergangene Zeit} \\
\end{aligned}
\end{equation}
\newline
\newline
Man kann die Formel natürlich verkürzen und die Animationdauer und -verzögerung direkt berechnen, aber so ist es verständlicher.

\newline
\newline
Hier sieht man den JavaScript Code:
\newline
\newline
\begin{lstlisting}[language=javascript,label={lst:JavaScript Timer}]
let currentEventBeginningTime = localStorage.getItem('currentEventBeginningTime');
let timePassed = (new Date() - new Date(currentEventBeginningTime + 'Z')) / 1000;
let animationDuration = ((new Date(currentEventEndTime + 'Z') - new Date(currentEventBeginningTime + 'Z')) / 1000);
firstHandSpan.style.animationDuration = animationDuration  + 's';
secondHandSpan.style.animationDuration = animationDuration + 's';
firstHandSpan.style.animationDelay = -timePassed + 's';
secondHandSpan.style.animationDelay = -timePassed  + 's';
\end{lstlisting}
\newline
\newline
Um die Animation während ihrer Laufzeit zu ändern, wird ein sogenannter Reflow ~\footcite{Reflow} erzwungen, indem die offsetWidth Eigenschaft abgefragt wird, die Animationsdauer und -verzögerung neu gesetzt wird und der Animationsname erst entfernt und dann wieder hinzugefügt wird.
\newline
\newline
