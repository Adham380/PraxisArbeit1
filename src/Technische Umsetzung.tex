%! Author = mboehme
%! Date = 23.02.2023

\subsection{Technische Umsetzung}
\subsubsection{Grobe Planung}
Das ganze wird als eine Azure App deployed. Diese lässt localhost Verbindungen zu und gibt einem die Möglichkeit das ganze als Single Page Application zu entwickeln, mit einem lokalen cookie cache für den eingeloggten Acccount. Somit ist keine Individualisierung notwendig.
\newline
Es wird lokal mithilfe von Dory-node.js gehostet.Das ganze wird als eine Azure App deployed. Diese lässt localhost Verbindungen zu und gibt einem die Möglichkeit die Web-Applikation wird als Single-Page-Application,mit einem lokalen cookie cache für den eingeloggten Acccount, entwickelt. Somit ist keine Individualisierung notwendig.
Zudem kann dann mit dem Phillips Display die Web-Applikation aufgerufen werden und als eine Art Model eines Model-View-Controller's verwendet werden.
\newline
Der Player braucht auf der OMS den Content-Typen "Web" mit der URL: "http://localhost:3000/content". Diese Seite wird dann lokal vom Player aufgerufen, worauf der Dory-nodejs Server antwortet und die tatsächliche Seite anbietet, die lokal auf dem Display liegt. So ist die URL immer die Gleiche, auf allen Geräten, damit Microsoft's redirect URI Bedingungen erfüllt werden können und Webserver kosten eingespart werden können, da Updates eher selten passieren sollten.
\newline
Zudem ist es auch für die Sicherheit der Nutzer so viel besser, da wir deshalb keinen Zugriff auf ihre Daten haben.
Zeiten müssen ISO 8601 Konform sein bei jeglichen API Anfragen an die Microsoft Graph API.
\newline
\pagebreak
\subsubsection{Hardware und Software}
Hardware:
\begin{itemize}
    \item Philips 10BDL4551T/00
\end{itemize}
\newline
Die Hardware ist ein Philips 10BDL4551T/00 Display.
Dieses Display ist ein 10 Zoll Touchscreen.
Es hat eine Auflösung von 1280x800 Pixeln und benutzt Android.
Dieses Display ist für Digital Signage gedacht und kann so auch als solches verwendet werden.
Die eingebauten RGB LED Strips können per SICP angesteuert werden.
\newline
\newline
Software:
\begin{itemize}
    \item Vue.js 3.x
    \item Microsoft Graph API per NPM Package
    \item Dory-Node.js
    \item Azure App
    \item Babel
    \item Webpack
    \item Node.js
    \item NPM
    \item Git
    \item EsLint
    \item IntelliJ IDEA
\end{itemize}
\newline
Vue.js ist eine JavaScript Framework, welches darauf spezialisiert ist,  Single Page Applikationen zu entwickeln.
\newline
\newline
Dory-Node.js ist ein Node.js Server, der es ermöglicht, eine Web-Applikation lokal zu hosten.
Dies wird benötigt, um einerseits die Raumbuchungsseite lokal anzubieten, damit die absoluten Redirect-URI Bedingungen von Microsoft und oAuth 2.0 erfüllt werden können und andererseits, um die Anwendung auf dem Display zu hosten und als Schnittstelle zwischen der Seite und dem den LED Strips des Displays zu agieren.
\newline
\newline
Die Microsoft Graph API wird per NPM Package verwendet, um die Anfragen an die Microsoft Graph API zu vereinfachen und den User einzuloggen.
\newline
\newline
Babel ist ein JavaScript Compiler, der es ermöglicht, moderne JavaScript Features zu verwenden, die von älteren Browsern nicht unterstützt werden, um so die Kompatibilität zu erhöhen.
\newline
\newline
Webpack ist ein JavaScript Bundler, der es ermöglicht, mehrere JavaScript Dateien zu einer einzigen zusammenzufassen, um so die Ladezeiten zu verkürzen.
\newline
\newline
Node.js ist ein JavaScript Runtime, der es ermöglicht, JavaScript Code auszuführen, ohne einen Browser zu benötigen.
Es ist das sogenannte Backend dieser Anwendung.
Dieser wird per Dory-Node.js gehostet.
\newline
\newline
NPM ist ein Package Manager, der es ermöglicht, JavaScript Packages zu installieren und zu verwalten.
\newline
\newline
Git ist ein Versionskontrollsystem, das es ermöglicht, Änderungen an Dateien zu verfolgen und zu verwalten.
\newline
\newline
EsLint ist ein Linter, der es ermöglicht, JavaScript Code zu analysieren und zu formatieren.
Da Testfälle bei JavaScript Projekten nicht immer vollständig möglich sind, ist es wichtig, dass der Code einheitlich ist und keine Fehler enthält.
Deshalb achtet EsLint auf die Einhaltung von Regeln, damit erst gar keine Fehler entstehen, die der JavaScript Compiler nicht erkennen kann.
Beispielsweise wird überprüft, ob Variablen deklariert wurden, bevor sie verwendet werden und anders herum.
\newline
\newline
IntelliJ IDEA ist eine IDE, die es ermöglicht, JavaScript Code zu schreiben und zu debuggen.
\newline
\newline
