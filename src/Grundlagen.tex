%! Author = mboehme
%! Date = 23.02.2023

\section{Grundlagen}\label{sec:grundlagen}
Um die Arbeit in vollem Umfang zu verstehen, ist es wichtig, dass die Grundlagen von Webanwendungen und Office365 verstanden werden.
\subsection{Webanwendungen}\label{subsec:webanwendungen}

\subsubsection{Was ist eine Webanwendung?}
Eine Webanwendung ist eine Anwendung, die über einen Browser aufgerufen werden.
Manche Webanwendungen sind auch Offline nutzbar.
\subsubsection{Vor- und Nachteile von Webanwendungen}
Eine Webanwendung ermöglicht es, dass die Anwendung von jedem Endgerät, welches über eine Internetverbindung verfügt, aus aufgerufen werden kann.
%Sie sind es nicht, sie können es sein
Webanwendungen sind effizient, benötigen wenig Speicherplatz, sind heutzutage gut abgesichert und aufgrund von der Tatsache, dass vieles standardisiert ist, gut wartbar.
\newline
\newline
Ein Nachteil von Webanwendungen ist, dass nicht alle Offline nutzbar sind.
%Einfacher gestalten, pls. Simple Sätze
Sie sind zudem nicht so schnell wie manche Desktopanwendungen und heutzutage auf die Verwendung von ~\cite{JavaScript} und unterschiedlichen Frameworks, die damit verbunden sind, angewiesen, welche oftmals umfangreich sind und Dependenzen haben, die weiterhin gepflegt werden müssen.
Dies führt oftmals dazu, dass falls eine Abhängigkeit nicht mehr gepflegt wird und andere geupdatet werden müssen und die Anwendung eventuell nicht mehr funktioniert.
Javascript lässt Laufzeitfehler zu.
%Das sind Glossareinträge. Fixen
\subsubsection{Was ist JavaScript?}
JavaScript ist eine Skriptsprache, die auf ECMAScript basiert.
\subsubsection{Was ist ECMAScript?}
ECMAScript ist eine Skriptsprache, die von der European Computer Manufacturers Association (ECMA) entwickelt wurde.
\subsubsection{Was ist Node.js?}
Node.js ist eine JavaScript Laufzeitumgebung, die auf dem V8 JavaScript Engine von Google basiert.
%\subsubsection{Was ist ein Framework?}
%Ein Framework ist eine Sammlung von Bibliotheken, die es ermöglichen, schneller und einfacher eine Anwendung zu entwickeln.
%\subsubsection{Was ist eine Bibliothek?}
%Eine Bibliothek ist eine Sammlung von Funktionen, die es ermöglichen, schneller und einfacher eine Anwendung zu entwickeln.
%\subsubsection{Was ist ein Package Manager?}
%Ein Package Manager ist ein Programm, das es ermöglicht, Bibliotheken und Frameworks zu installieren und zu verwalten.
%\subsubsection{Was ist ein Package?}
%Ein Package ist eine Sammlung von Bibliotheken und Frameworks, die es ermöglichen, schneller und einfacher eine Anwendung zu entwickeln.
%\subsubsection{Was ist ein Build Tool?}
%Ein Build Tool ist ein Programm, das es ermöglicht, eine Anwendung zu kompilieren und zu verpacken.
%\subsubsection{Was ist ein Compiler?}
%Ein Compiler ist ein Programm, das es ermöglicht, eine Anwendung zu kompilieren.
%\subsubsection{Was ist ein Transpiler?}
%Ein Transpiler ist ein Programm, das es ermöglicht, eine Anwendung zu transpilieren.
%\subsubsection{Was ist eine Transpilation?}
%Eine Transpilation ist ein Prozess, bei dem eine Sprache in eine andere Sprache übersetzt wird.
%\subsubsection{Was ist eine Kompilation?}
%Eine Kompilation ist ein Prozess, bei dem eine Sprache in Maschinencode übersetzt wird.
%\subsubsection{Was ist ein Bundler?}
%Ein Bundler ist ein Programm, das es ermöglicht, mehrere Dateien zu einer einzigen Datei zusammenzufassen.
%\subsubsection{Was ist ein Linter?}
%Ein Linter ist ein Programm, das es ermöglicht, die Codequalität zu überprüfen.

%Erklären was die Microfosft Graph API ist. Nicht mehr. Aufzählung etc. weg
\subsection{Microsoft Graph API}\label{subsec:microsoft-graph-api}
Die Microsoft Graph API ist eine \gls{RESTful} web API, die es einem erlaubt auf Daten von Microsoft 365 und Office 365 zuzugreifen.
\newglossaryentry{RESTful}{
    name=RESTful,
    description={RESTful ist ein Synonym für Representational State Transfer. RESTful ist ein Architekturstil für die Entwicklung von Webdiensten.}
}
Mithilfe dieser API wurde das Projekt letztendlich umgesetzt.
Weitere standen jedoch zur Verfügung:
\begin{itemize}
    \item Microsoft Outlook API
    \item Microsoft Exchange API
    \item Microsoft SharePoint API
    \item Microsoft OneDrive API
    \item Microsoft Teams API
    \item Microsoft Power Automate
\end{itemize}
Einige dieser APIs, sind nur für bestimmte Microsoft 365 und Office 365 Abonnements verfügbar.
Die Microsoft Graph API ist jedoch, derzeit, für alle Abonnements verfügbar.
Zudem ist die Microsoft Graph API die einzige API, die es einem erlaubt auf alle Daten von Microsoft 365 und Office 365 zuzugreifen, da sie die meisten anderen APIs integriert.
Der wichtigste Faktor bei der Entscheidung war es jedoch, dass die Microsoft Graph API, mithilfe von Azure AD, die Authentifizierung und Autorisierung von Benutzern erlaubt.
Dies ist für die Anwendung von großer Bedeutung, da es dem Benutzer ermöglicht sich mit seinem Microsoft 365 oder Office 365 Account anzumelden und somit auf seine Daten zuzugreifen.
%Gehört nicht hier hin, sondern Vorgehensweise

