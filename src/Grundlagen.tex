%! Author = mboehme
%! Date = 23.02.2023

\section{Grundlagen}
Um die Arbeit in vollem Umfang zu verstehen, ist es wichtig, dass die Grundlagen von Webanwendungen und Office365 verstanden werden.
\subsection{Webanwendungen}
\subsubsection{Was ist eine Webanwendung?}
Eine Webanwendung ist eine Anwendung, die über das Internet aufgerufen wird.
\subsubsection{Vor- und Nachteile von Webanwendungen}
Eine Webanwendung ermöglicht es, dass die Anwendung von jedem Endgerät aus aufgerufen werden kann, welches über eine Internetverbindung verfügt.
Webanwendungen sind relativ effizient, benötigen wenig Speicherplatz, heutzutage sehr sicher und aufgrund von der Tatsache, dass vieles standardisiert ist, gut wartbar.
\newline
\newline
Ein Nachteil von Webanwendungen ist, dass nicht alle Offline nutzbar sind.
Sie sind zudem nicht so schnell wie manche Desktopanwendungen und sind heutzutage auf die Verwendung von
\cite{JavaScript}
und all den Frameworks, die damit verbunden sind, angewiesen, welche oftmals sehr umfangreich sind und Dependenzen haben, die weiterhin gepflegt werden müssen.
Dies führt oftmals dazu, dass falls eine Abhängigkeit nicht mehr gepflegt wird und andere geupdatet werden müssen, die Anwendung eventuell nicht mehr funktioniert.
Zudem ist die JavaScript Sprache nicht immer so fehlerabfangend wie andere Sprachen und lässt Kompilierungen zu, die dann im Laufzeitfehler enden.
\subsubsection{Was ist JavaScript?}
JavaScript ist eine Skriptsprache, die auf ECMAScript basiert.
\subsubsection{Was ist ECMAScript?}
ECMAScript ist eine Skriptsprache, die von der European Computer Manufacturers Association (ECMA) entwickelt wurde.
\subsubsection{Was ist Node.js?}
Node.js ist eine JavaScript Laufzeitumgebung, die auf dem V8 JavaScript Engine von Google basiert.
\subsubsection{Was ist ein Framework?}
Ein Framework ist eine Sammlung von Bibliotheken, die es ermöglichen, schneller und einfacher eine Anwendung zu entwickeln.
\subsubsection{Was ist eine Bibliothek?}
Eine Bibliothek ist eine Sammlung von Funktionen, die es ermöglichen, schneller und einfacher eine Anwendung zu entwickeln.
\subsubsection{Was ist ein Package Manager?}
Ein Package Manager ist ein Programm, das es ermöglicht, Bibliotheken und Frameworks zu installieren und zu verwalten.
\subsubsection{Was ist ein Package?}
Ein Package ist eine Sammlung von Bibliotheken und Frameworks, die es ermöglichen, schneller und einfacher eine Anwendung zu entwickeln.
\subsubsection{Was ist ein Build Tool?}
Ein Build Tool ist ein Programm, das es ermöglicht, eine Anwendung zu kompilieren und zu verpacken.
\subsubsection{Was ist ein Compiler?}
Ein Compiler ist ein Programm, das es ermöglicht, eine Anwendung zu kompilieren.
\subsubsection{Was ist ein Transpiler?}
Ein Transpiler ist ein Programm, das es ermöglicht, eine Anwendung zu transpilieren.
\subsubsection{Was ist eine Transpilation?}
Eine Transpilation ist ein Prozess, bei dem eine Sprache in eine andere Sprache übersetzt wird.
\subsubsection{Was ist eine Kompilation?}
Eine Kompilation ist ein Prozess, bei dem eine Sprache in Maschinencode übersetzt wird.
\subsubsection{Was ist ein Bundler?}
Ein Bundler ist ein Programm, das es ermöglicht, mehrere Dateien zu einer einzigen Datei zusammenzufassen.
\subsubsection{Was ist ein Linter?}
Ein Linter ist ein Programm, das es ermöglicht, die Codequalität zu überprüfen.
