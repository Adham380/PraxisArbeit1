%! Author = mboehme
%! Date = 23.02.2023
\section{Grundlagen}\label{sec:grundlagen}
Um die Arbeit in vollem Umfang zu verstehen, ist es wichtig, dass die Grundlagen von Webanwendungen verstanden sind.
\subsection{Webanwendungen}\label{subsec:webanwendungen}

\subsubsection{Was ist eine Webanwendung?}
Eine Webanwendung ist eine Anwendung, die über einen Browser aufgerufen werden.
Manche Webanwendungen sind auch Offline nutzbar.
\subsubsection{Vor- und Nachteile von Webanwendungen}
Eine Webanwendung ermöglicht es, dass die Anwendung von jedem Endgerät, welches über eine Internetverbindung verfügt, aus aufgerufen werden kann.
Webanwendungen können effizient, benötigen im Optimalfall wenig Speicherplatz, und können gut abgesichert werden.
Aufgrund der Tatsache, dass Browser gut standardisiert sind, können Webanwendungen schnell entwickelt werden.
\newline
\newline
Ein Nachteil von Webanwendungen ist, dass nicht alle Offline nutzbar sind.
Sie sind zudem nicht so schnell wie manche Desktopanwendungen, die auf mehr Rechenleistung zugreifen können, da sie im Regelfall Programmiersprachen nutzen, die näher an Maschinensprache sind.
\newglossaryentry{Native Sprache}{name={Native Sprache}, description={Eine Sprache, die nicht erst übersetzt werden muss, sondern direkt ausgeführt werden kann.}}
Webanwendungen besitzen heutzutage ausschließlich~\cite{JavaScript} als native Sprache.
\newglossaryentry{Frameworks}{name={Frameworks}, description={Eine Sammlung von Bibliotheken, die es ermöglichen, schneller und einfacher eine Anwendung zu entwickeln.}}
Um mehr Flexibilität und Funktionalität, ohne großen Aufwand zu ermöglichen, sind Webanwendungen auf~\gls{Frameworks}, angewiesen, welche oftmals umfangreich sind und Dependenzen besitzen, die weiterhin gepflegt werden müssen.
Dies führt oftmals dazu, dass falls eine Abhängigkeit nicht mehr gepflegt wird oder andere geupdatet werden müssen, die Anwendung eventuell nicht mehr funktioniert, falls die Abhängigkeit nicht mehr kompatibel ist.
Außerdem lässt JavaScript Laufzeitfehler zu.
%Das sind Glossareinträge. Fixen
%\subsubsection{Was ist ein Framework?}
%Ein Framework ist eine Sammlung von Bibliotheken, die es ermöglichen, schneller und einfacher eine Anwendung zu entwickeln.
%\subsubsection{Was ist eine Bibliothek?}
%Eine Bibliothek ist eine Sammlung von Funktionen, die es ermöglichen, schneller und einfacher eine Anwendung zu entwickeln.
%\subsubsection{Was ist ein Package Manager?}
%Ein Package Manager ist ein Programm, das es ermöglicht, Bibliotheken und Frameworks zu installieren und zu verwalten.
%\subsubsection{Was ist ein Package?}
%Ein Package ist eine Sammlung von Bibliotheken und Frameworks, die es ermöglichen, schneller und einfacher eine Anwendung zu entwickeln.
%\subsubsection{Was ist ein Build Tool?}
%Ein Build Tool ist ein Programm, das es ermöglicht, eine Anwendung zu kompilieren und zu verpacken.
%\subsubsection{Was ist ein Compiler?}
%Ein Compiler ist ein Programm, das es ermöglicht, eine Anwendung zu kompilieren.
%\subsubsection{Was ist ein Transpiler?}
%Ein Transpiler ist ein Programm, das es ermöglicht, eine Anwendung zu transpilieren.
%\subsubsection{Was ist eine Transpilation?}
%Eine Transpilation ist ein Prozess, bei dem eine Sprache in eine andere Sprache übersetzt wird.
%\subsubsection{Was ist eine Kompilation?}
%Eine Kompilation ist ein Prozess, bei dem eine Sprache in Maschinencode übersetzt wird.
%\subsubsection{Was ist ein Bundler?}
%Ein Bundler ist ein Programm, das es ermöglicht, mehrere Dateien zu einer einzigen Datei zusammenzufassen.
%\subsubsection{Was ist ein Linter?}
%Ein Linter ist ein Programm, das es ermöglicht, die Codequalität zu überprüfen.

%Erklären was die Microfosft Graph API ist. Nicht mehr. Aufzählung etc. weg
\subsection{Microsoft Graph API}\label{subsec:microsoft-graph-api}
\newglossaryentry{API}{name={API}, description={Eine Schnittstelle, die es ermöglicht, auf Daten zuzugreifen.}}
Die Microsoft Graph API ist eine \gls{RESTful} web~\gls{API}, die es einem erlaubt auf Daten von Microsoft 365 und Office 365 zuzugreifen.
\newglossaryentry{RESTful}{
    name=RESTful,
    description={REST ist ein Synonym für Representational State Transfer. RESTful ist ein Architekturstil für die Entwicklung von Webdiensten.}
}

Es gibt einige APIs, die sich auf bestimmte Microsoft 365 und Office 365 Abonnements beziehen, wie beispielsweise die Microsoft Teams API\@.
Die Microsoft Graph vereint jedoch die meisten APIs, die Microsoft 365 und Office 365 betreffen, in einer einzigen API\@.
Dies ermöglicht Zugang auf Daten von Microsoft 365 und Office 365, wie beispielsweise Benutzer, Kalender, E-Mails, Dateien, etc.


