%! Author = mboehme
%! Date = 28.03.2023

% Preamble
\subsection{Optimierung}\label{subsec:optimierung}
\subsubsection{Abfrageoptimierung}\label{subsubsec:abfrageoptimierung}
Mithilfe der oDatav4 (~\cite{oData}) API wurden die Daten aus der jeweiligen Microsoft Datenbank im Voraus gefiltert, um so nicht notwendige Daten zu vermeiden und die Performance drastisch zu erhöhen, als auch Datenvolumen zu sparen.
Die oData v4 API verhält sich dabei wie eine SQL Datenbank, wobei die Daten in Tabellen gespeichert sind.
Es wurden die Klauseln \("\$\)filter\("\) und \("\$\)top\("\) verwendet, um nur Termine für den aktuellen und nächsten Tag zu erhalten und dies einzuschränken auf die top 300 Termine, falls jemand versucht das System zu überlasten.
Eine Sortierung ist nicht notwendig, da die Termine bereits nach Startzeit sortiert sind.
Hier sieht man die oData v4 API Anfrage, die an die Microsoft Graph API gesendet wird:
\newline
%add Fix code listing and remove .
\begin{lstlisting}[language=JavaScript,label={lst:JavaScript oData v4 API Anfrage}]
     let url =  "https://graph.microsoft.com/v1.0/me/findMeetingTimes/?$filter=start/dateTime" +  "ge"  + "${todayDate} and end/dateTime le ${tomorrowDate}&$top=300";
\end{lstlisting}
\("\)https://graph.microsoft.com/v1.0/me/findMeetingTimes/\("\) ist die URL der Microsoft Graph API.
Des Weiteren kommuniziert sie, welche API, spezifisch angesprochen werden soll.
Da die Termine für den aktuellen und nächsten Tag benötigt werden, wird die \("\)FindMeetingTimes\("\) verwendet, welche die Kalender API verwendet.
Mit dem \("\)me\("\) wird der aktuelle Benutzer mitgegeben.
Dieser wird automatisch vom Microsoft Graph erkannt.
\newglossaryentry{Query Parameter}{name=Query Parameter,description={Query Parameter sind Parameter, die an eine URL angehängt werden können, um die Anfrage zu verfeinern.}}
Ab dem Fragezeichen werden die Query Parameter übergeben.
Die Antwort der Anfrage, im Teil der URL, vor dem Fragezeichen, werden durch die Query Parameter auf die benötigten Daten gefiltert.
\newline
Mit \("\)start/dateTime\("\) + \("\)ge\("\) + \("\){todayDate}\("\) wird der Startzeitpunkt des Termins mit dem heutigen Datum verglichen.
Es wird geprüft, ob der Startzeitpunkt des Termins nach dem heutigen Datum liegt.
Das \("\)ge\("\) steht für \("\)greater than or equal to\("\) und bedeutet, dass der Startzeitpunkt des Termins nach oder gleich dem heutigen Datum liegt.
Dies wird eigentlich für mathematische Vergleiche verwendet, aber es funktioniert auch für Strings, da jeder Charakter einen ASCII Wert hat.
\newline
Somit erhalten wir nur Daten zurück, die heute oder in der Zukunft liegen.
\newline
Mit \("\)end/dateTime\("\) + \("\)le\("\) + \("\) {tomorrowDate}\("\) wird der Endzeitpunkt des Termins mit dem morgigen Datum verglichen.
Letzten Endes erhalten wir nur Termine zurück, die heute oder morgen stattfinden.
Die Verarbeitung dieser Daten erfolgt ISO 8601 konform.
Sobald diese jedoch angezeigt werden, wird darauf geachtet, dass es die lokale Zeitzone ist.
\newline
