%! Author = mboehme
%! Date = 23.02.2023

% Preamble
%Problemanalyse davor. (Zusammenfassung)
%Dies kommt in den Lösungsentwurf
\section{Abgrenzung zu ähnlichen Produkten}\label{sec:abgrenzung-zu-ahnlichen-produkten}
Um die Anwendung von anderen Anwendungen abzugrenzen, werden diese tabellarisch verglichen.
\newline
\newline
%\subsection{Microsoft Teams}\label{subsec:microsoft-teams}
%Microsoft Teams ermöglicht Terminbuchungen genauso, erlaubt jedoch zu detaillierte Eingriffe in das Konto, welcher für den Raum vorgesehen ist.
%Einstellungen, die nicht verändert werden sollten, können dann erreicht werden.
%Beispielsweise könnte ein Benutzer, der Terminbuchungsrechte für den Kalender des Raumes erhält, Einstellungen wie die Sichtbarkeit, Pufferzeiten oder die Anzahl der Termine pro Tag verändern.
%Dies ist jedoch nicht gewünscht, da diese Einstellungen nur vom Administrator des Kalenders verändert werden sollten.
%Nichtsdestotrotz muss ein Benutzer in der Lage sein, den Kalender einzusehen, um zu sehen, ob der Raum belegt ist und diesen zu buchen.
%\newline
%\subsection{Outlook}\label{subsec:outlook}
%Outlook ist ein E-Mail-Programm, welches ebenfalls Terminbuchungen ermöglicht.
%Outlook ist nicht für die Terminbuchung in Konferenzräumen gedacht und kann je nach E-Mail-Konto-Typ nicht in vollem Umfang genutzt werden.
%Beispielsweise unterscheiden sich Synchronisationsintervalle zwischen Konten-Typen.
%Auch hier können Einstellungen, die nicht verändert werden sollten, erreicht werden und Kalender eingesehen werden, die nur für Administratoren gedacht sind.
%\newline
%\subsection{Microsoft Bookings}\label{subsec:microsoft-bookings}
%Microsoft Bookings ist eine weitere Applikation, die Terminbuchungen ermöglicht.
%Sie wird hauptsächlich für Terminbuchungen in beispielsweise Friseursalons, Fitnessstudios oder ähnlichen Geschäften genutzt.
%Dies ist jedoch der falsche Anwendungsfall für die Terminbuchung in Konferenzräumen.
%Auch die Einschränkungen und Freiheiten, die diese Applikation bietet, sind hier nicht passend.
%Microsoft Bookings erlaubt dem Anwender, eine vorgefertigte Webseite zu erstellen, auf der ein Kunde Termine buchen kann.
%Sie ist nicht für internes Management gedacht und lässt wenig Anpassungen zu.
%Beispielsweise wird nicht der aktuelle Raumstatus angezeigt, sondern nur die Anzahl der verfügbaren Termine.
%%Einscrhänkungen und Freiheiten erläutern
%\newline
%\subsection{Microsoft Power Automate}\label{subsec:microsoft-power-automate}
%Microsoft Power Automate ist eine Applikation, die es ermöglicht, Workflows zu erstellen.
%Diese Workflows können dann automatisiert ausgeführt werden.
%Diese kann auch ähnliche Funktionen wie die Applikation, die in dieser Arbeit entwickelt wird, bieten.
%Jedoch ist die Applikation nur für innerhalb der Organisationen vorgesehen und nicht für Kunden.
%erläutern
%Warum pagebreak?

%Make a table that shows the differences between the products and our solution
\small
\begin{tabularx}{\textwidth}{|X|X|X|X|X|X|}
    \hline
    \textbf{Anforderungen} & \textbf{Microsoft Teams} & \textbf{Microsoft Outlook}& \textbf{Microsoft Bookings} & \textbf{Microsoft Power Automate} & \textbf{Gewichtung} \\
    \hline
    Webapplikation & \cmark & \cmark & \cmark & \cmark & 2 \\
    \hline
    Keine externe Registrierung & \cmark & \cmark & \cmark & \cmark & 2 \\
    \hline
    Gastgeber- und Teilnehmerlogos & \xmark & \xmark & \xmark & \xmark & 2 \\
    \hline
    Zwei-Tages-Ansicht & \cmark & \cmark & \xmark & \cmark & 1  \\
    \hline
    Anzeige aller verfügbaren Termine & \xmark & \xmark & \xmark & \cmark & 2 \\
    \hline
    Anzeige des jetzigen Raumstatus & \cmark & \cmark & \xmark & \cmark & 2 \\
    \hline
    Minimale Anzahl an Klicks & \xmark & \cmark & \xmark & \cmark & 1 \\
    \hline
    Minimale Benutzeroberfläche & \xmark & \xmark & \xmark & \xmark & 1  \\
    \hline
    Funktionen auf ein Minimum beschränken & \xmark & \xmark & \xmark & \cmark & 1  \\
    \hline
    Angemessene voreingestellte Softwaredistribution & \xmark & \xmark & \xmark & \xmark & 2 \\
    \hline
\end{tabularx}
\normalsize
\newline
\newline
\newline
Die maximal mögliche Punktzahl ist 16.
Microsoft Teams erreicht 6 von 16 Punkten.
Microsoft Outlook erreicht 8 von 16 Punkten.
Microsoft Bookings erreicht 7 von 16 Punkten.
Microsoft Power Automate erreicht 11 von 16 Punkten.
\newline
Keins der Produkte erfüllt alle Anforderungen.
Microsoft Power Automate erfüllt die meisten Anforderungen.
Eine ideale Softwarelösung sollte alle Anforderungen erfüllen.
Die mit \textbf{2} gewichteten Anforderungen sind Kernanforderungen, die erfüllt werden müssen.
Weitere Anforderungen, die nicht erfüllt werden, sind nicht kritisch, aber wären dennoch wünschenswert.
Nichtsdestotrotz erffült kein Produkt die Kernanforderungen in vollem Umfang erfüllt.
\newline
\newline
\subsection{Spezifische Gründe für die Entwicklung einer eigenen Applikation}\label{subsec:spezifische-grunde-fur-die-entwicklung-einer-eigenen-applikation}
Wenn ein Kunde bei der DOOH media GmbH einkauft, kauft er im Regelfall ein umfangreiches Softwarepaket.
\newglossaryentry{Kiosk-Modus}{name=Kiosk-Modus, description={Der Kiosk-Modus ist ein Modus, in dem ein Gerät, wie beispielsweise ein Tablet, nur eine Applikation ausführen kann.}}
Dieses Paket inkludiert die Möglichkeit die Geräte in einen sogenannten~\gls{Kiosk-Modus} zu versetzen.
Dafür haben wir eine eigene Applikation, die Oxygen Player App, entwickelt.
Zudem haben wir eine \("\)OMS\("\)-Lösung (Oxygen Media Server), die es ermöglicht, die Inhalte auf den Geräten zu verwalten.
Falls der Kunde also zu bestimmten Zeiten eigene Inhalte oder Inhalte von externen Anbietern, auf den Geräten, darstellen möchte, kann er dies über die \("\)OMS\("\)-Lösung tun.
Diese Inhalte werden dann auf den Geräten abgespielt.
\newline
Die Idee ist hier, dass der Kunde nicht an die Terminbuchungs-Webapplikation gebunden ist, sondern diese Applikation nur als zusätzliche Funktion angeboten bekommt.
Viele der Softwarelösungen, die schon für genau diesen Anwendungsfall existieren, beinhalten Bindungen an andere Softwarelösungen, die der Kunde nicht benötigt.
Sie machen es dem Kunden schwer, die Softwarelösung zu nutzen, da er sich mit anderen Softwarelösungen auseinandersetzen muss, die er nicht benötigt und die gegebenenfalls nicht in einem Kiosk-Modus laufen können oder parallel zu unserer restlichen Software, die der Kunde haben möchte.
Außerdem erfordern einige dieser Softwarelösungen eine Installation auf dem Gerät oder erzwingen Konten-Registrierungen, die der Kunde nicht benötigt.
\newline
%Gehört in den Soll-Zustand
Die Alternativprodukte, die in den vorherigen Abschnitten beschrieben wurden, erfüllen nicht alle Anforderungen, die der Kunde an die Softwarelösung hat.
Alles wird lokal auf dem Gerät abgespielt und es wird keine Installation auf dem Gerät benötigt.
Es wird keine Registrierung eines Kontos benötigt und niemand kann die Daten von außerhalb, außer natürlich Microsoft, einsehen, da wirklich alles lokal gehostet wird.
Die Inhalte werden dynamisch auf dem Gerät geladen und verarbeitet.
