%! Author = Adham Aijou
%! Date = 20/02/2023
% Preamble
\documentclass[11pt]{article}
\usepackage{hyperref}
\usepackage{mdframed}
\usepackage[ngerman]{babel}
\usepackage[T1]{fontenc}
\usepackage[utf8]{inputenc}
\usepackage[paper=a4paper,left=40mm,right=20mm,top=25mm,bottom=20mm,footskip=25pt ]{geometry}
\usepackage[backend=biber,style=alphabetic, uniquename=false]{biblatex}
\usepackage{mathptmx} %Font Times New Roman
\usepackage[acronyms, toc,section,numberedsection = autolabel,automake,]{glossaries} %Erstellt ein Abkürzungsverzeichnis
\usepackage{array}
\usepackage{wrapfig} %Bilder können fließend in den Text eingebunden werden
\usepackage{xcolor}
\usepackage{filecontents}
\usepackage{sectsty}
\usepackage{titletoc}
\usepackage{tocvsec2}
\usepackage[title,titletoc]{appendix}
\usepackage{enumitem}
\usepackage{pgfplots}
\usepackage{tikz}
\usepackage{longtable} % Tabellen können Seitenumbrüche enthalten
\usepackage{listings} % Listings für Codebeispiele
\usepackage{pdfpages} % PDFs einbinden
\lstdefinestyle{JavaScript}
{
    language=JavaScript,
    basicstyle=\footnotesize\ttfamily,
    keywordstyle=\color{blue}\bfseries,
    commentstyle=\color{green},
    stringstyle=\color{red},
    breaklines=true,
    breakatwhitespace=true,
    tabsize=2,
    captionpos=b,
    keepspaces=true,
    numbers=right,
    numberstyle=\tiny,
    stepnumber=1,
    numbersep=5pt,
    showspaces=false,
    showstringspaces=false,
    showtabs=false,
    frame=single,
    rulecolor=\color{black!20},
    backgroundcolor=\color{white},
    literate={ä}{{\"a}}1
    {ö}{{\"o}}1
    {ü}{{\"u}}1
    {Ä}{{\"A}}1
    {Ö}{{\"O}}1
    {Ü}{{\"U}}1
    {ß}{{\ss}}1
    {~}{{\textasciitilde}}1
}
\lstset{basicstyle=\footnotesize,
    inputencoding={utf8},
    extendedchars=false,
    language={Bash},
    escapeinside=``,
    frame = single}
\lstset{literate=%
    {Ö}{{\"O}}1
    {Ä}{{\"A}}1
    {Ü}{{\"U}}1
    {ß}{{\ss}}1
    {ü}{{\"u}}1
    {ä}{{\"a}}1
    {ö}{{\"o}}1
    {~}{{\textasciitilde}}1
}

\lstMakeShortInline[columns=fixed, basicstyle=\ttfamily\color{darkgray}]@
\usepackage{color}
\definecolor{lightgray}{rgb}{.9,.9,.9}
\definecolor{darkgray}{rgb}{.4,.4,.4}
\definecolor{purple}{rgb}{0.65, 0.12, 0.82}

\lstdefinelanguage{JavaScript}{
    keywords={typeof, new, true, false, catch, function, return, null, catch, switch, var, if, in, while, do, else, case, break},
    keywordstyle=\color{blue}\bfseries,
    ndkeywords={class, export, boolean, throw, implements, import, this},
    ndkeywordstyle=\color{darkgray}\bfseries,
    identifierstyle=\color{black},
    sensitive=false,
    comment=[l]{//},
    morecomment=[s]{/*}{*/},
    commentstyle=\color{purple}\ttfamily,
    stringstyle=\color{red}\ttfamily,
    morestring=[b]',
    morestring=[b]"
}

\lstset{
    language=JavaScript,
    backgroundcolor=\color{lightgray},
    extendedchars=true,
    basicstyle=\footnotesize\ttfamily,
    showstringspaces=false,
    showspaces=false,
    numbers=none,
    numberstyle=\footnotesize,
    numbersep=9pt,
    tabsize=2,
    breaklines=true,
    showtabs=false,
    captionpos=b
}

%\addtokomafont{disposition}{\rmfamily} %Serifenschrift für Überschriften
\counterwithout{footnote}{section} %Durchgehende Fußnoten Nummerierung
\setlength\parindent{0pt} % Einrückung für neue Absätze
\usepackage[font=footnotesize,justification = centering]{caption} %Font für Bildunterschriften
\usepackage{pifont}   % Wird für den Punkt neben den Seitzenzahlen benötigt
\usepackage[footsepline=1pt]{scrlayer-scrpage} % Anpassung von Kopf/Fußzeilen
\usepackage{setspace}
\usepackage{microtype}
\usepackage{tabularx}
\usepackage{booktabs}
\usepackage{ragged2e}
% Document
\makeglossaries
\defglsentryfmt{\color{black}\bfseries\glsgenentryfmt}
%\addbibresource{main.bib}
\addbibresource{literatur.bib}
\begin{document}
%    - Vorname vom Betreuer
%- Einleitung sollte Titel beinhalten
%- two footnotes on page 17 (Reflow)
%- Fragestellungen und Ziele nicht aufzählen, sondern schlussfolgern aus Problemstellung
%- Nicht begründen warum
%- Nicht Ziele/Fragen zuerst, erst Hinführung mit der Zusammenfassung
%- caniuse Glossar (Deutsch bitte)
%- 500ms Satz sollte dorthin, wo der Leser davon weiß
%- Die detaillierten Aspekte
%- Struktur der Arbeit nur in ein Kapitel
%- Tabellen maybe als Screenshot einfügen
    %! Author = Adham Aijou
%! Date = 20/02/2023

% Preamble
\documentclass[11pt]{article}

% Packages
\usepackage{amsmath}
\usepackage{graphicx}
\pagenumbering{Roman}
\begin{titlepage}
    \centering
    \includegraphics[width=0.35\textwidth]{Bilder/doohMedia}\par\vspace{1cm}
    \vspace{2cm}
    Fachhochschule für die Wirtschaft Hannover \\
    - FHDW -\\
    {\scshape\Large Praxisarbeit \par}
    \vspace{1cm}
    {\huge\bfseries Ressourcen- und Terminplanung in Office365 mit Hilfe von Microsoft Graph API \par}
    \vspace{3cm}

    \normalsize{
        \begin{tabular}{l l}
            \textbf{Adham:} & \textbf{Aijou} \\
            &	Brinker Straße 72 \\
            &	30851, Langenhagen \\
            \\

            \textbf{Mentor:} & \textbf{Prof. Dr.  Ing. Klinger}\\
            \\
            \textbf{Studiengruppe:} & \textbf{HFI421IN}\\
            \\
            \textbf{Matrikelnummer:}  & \textbf{600142} \\
            \\
            \textbf{Ausbildungsbetrieb:}  & \textbf{DOOH media GmbH}\\
            & Frankenring 18\\
            & 30855 Langenhagen    \\
            \\
            \textbf{Eingereicht am:} & xx.xx.xxxx
        \end{tabular}\\
    }
    \vfill
    \flushright \includegraphics[width=0.35\textwidth]{Bilder/doohMedia}\par\vspace{1cm}

\end{titlepage}
% Document
\begin{document}



\end{document}
    %! Author = mboehme
%! Date = 21.02.2023



% Document
    \tableofcontents
    \newpage



    %! Author = mboehme
%! Date = 21.02.2023

% Preamble

% Document

\section{Einleitung}\label{sec:einleitung}
Die Praxisarbeit befasst sich mit der Ressourcen- und Terminplanung in Office365 mithilfe von Microsoft Graph API~\cite{microsoftGraphApi}.
Die Arbeit ist in drei Teile gegliedert.
Im ersten Teil wird die Theorie der Ressourcen- und Terminplanung in Office365 mithilfe von Microsoft Graph API erläutert.
Im zweiten Teil wird die praktische Umsetzung der Theorie beschrieben.
Im dritten Teil wird die Arbeit abschließend bewertet.
    \newline
    \subsection{Fragestellungen der Arbeit}\label{subsec:fragestellungen-der-arbeit}
%Die Fragestellungen der Arbeit lauten:
%    \begin{itemize}
%        \item Wie funktioniert die Ressourcen- und Terminplanung in Office365 mithilfe von Microsoft Graph API?
%        \item Wie kann die Ressourcen- und Terminplanung in Office365 mithilfe von Microsoft Graph API praktisch umgesetzt werden?
%        \item Wie kann die Arbeit abschließend bewertet werden?
%    \end{itemize}
%Konkret bedeutet dies:
Inwiefern und wie sinnvoll, ist der Einsatz von Microsoft Graph API für die Ressourcen- und Terminplanung in Office365 und wie lässt sich dies in der Praxis umsetzen?
Dabei sollte berücksichtigt werden, dass dies auf die Anforderungen eines Kundenauftrags bezogen sind, welche im Kapitel~\ref{subsubsec:anforderungen} erläutert werden.
Deshalb stellt sich auch die Frage, ob die Implementierung sinnvoller ist, als weiterhin die bestehende Lösung (Outlook, Teams, etc.) zu nutzen.
Wie sollte sowas umgesetzt werden und welche Aspekte der Microsoft Graph API sind dafür relevant?
Als Letztes stellt sich die Frage, wie die Arbeit abschließend bewertet werden sollte.

\subsection{Ziele der Arbeit}\label{subsec:ziele-der-arbeit}
Sowohl die theoretische als auch die praktische Umsetzung der Theorie, müssen in der Arbeit beschrieben werden.
Abschließend bedarf es sowohl einer objektiven als auch einer subjektiven Bewertung der Arbeit, um festzustellen, ob die Implementierung dieser API, für den Kunden sinnvoll ist.
Die objektive Bewertung sollte qualitative und quantitative Aspekte der Arbeit berücksichtigen.
    \subsection{Ergebnisse der Arbeit}\label{subsec:ergebnisse-der-arbeit}
Die Arbeit hat alle Ziele erreicht.
    \newglossaryentry{SPA}{name=Single Page Application (SPA), description={Eine Single Page Application (SPA) ist eine Webanwendung, die nur eine HTML-Seite besitzt.
    Diese Seite wird beim Laden der Anwendung geladen und bleibt während der gesamten Nutzung der Anwendung bestehen.}, first={Single Page Application (SPA)}, text={SPA}, short=SPA}
%\newglossaryentry{SPA}{name=SPA, description={Eine Single Page Application (SPA) ist eine Webanwendung, die nur eine HTML-Seite besitzt.}}
%\newacronym{SPA}{SPA}{Single Page Application}
    Die Microsoft Graph API ist erfolgreich eingesetzt und die Ressourcen- und Terminplanung in Office365, mithilfe von der Microsoft Graph API, wurde erfolgreich als~\gls{SPA} umgesetzt.
\newglossaryentry{TBT}{name={Total-Blocking-Time (TBT)}, description= {Die Total-Blocking-Time (TBT) ist eine Metrik, die die Gesamtzeit misst, die eine Seite blockiert wird, bis sie interaktiv ist.},
first={Total-Blocking-Time (TBT)}, text={TBT}}

Die detaillierten Aspekte Ergebnisse der Arbeit sind in den Kapiteln~\ref{sec:ergebnis} und ~\ref{sec:technische-umsetzung} beschrieben.
Das ausformulierte Ergebnis und die Schlussfolgerung dieser, sind in Kapitel~\ref{sec:ergebnis} erläutert.
\newpage


    %! Author = mboehme
%! Date = 23.02.2023

% Preamble
\subsection{Zusammenfassung}
Die Praxisarbeit wurde durch einen Kundenauftrag ins Leben gerufen.
Es sei umständlich für gewisse Kunden für Konferenzräume Termine zu buchen.
Jedes Mal muss nämlich ein Mitarbeiter die Verfügbarkeit eines Konferenzraumes auf einer Applikation mit zu vielen Zwischenschritten prüfen und dann einen Termin buchen.
Dieser Prozess ist sehr zeitaufwendig und kann zu Fehlern führen.
Falls beispielsweise 30 Räume zur Verfügung stehen und der Mitarbeiter ausschließlich diesen einen Raum auf Verfügbarkeit prüfen möchte, ist dies einerseits umständlich und andererseits gegebenenfalls durch den Administrator eingeschränkt.
Wenn der Administrator dann den Raum für den Kunden freigibt, muss der Mitarbeiter den Raum nochmal auf Verfügbarkeit prüfen, bevor er den Termin bucht und falls zu viele Rechte freigegeben werden, kann es zu Fehlern kommen,
die dann von einem Administrator behoben werden müssen.
Zudem hätten dann Mitarbeiter Rechte, die sie nicht benötigen und könnten außerhalb ihrer Zuständigkeit, Fehler machen.
\newline
Daher soll eine Applikation entwickelt werden, die es ermöglichen soll, dass der Mitarbeiter selbstständig einen Termin buchen, oder einsehen kann, wann ein Termin denn stattfinden soll, wenn er sowieso schon im Gebäude und sich vor dem Raum befindet.
\newline
Bei diesem Kunden sind viele Mitarbeiter nicht aus der gleichen Abteilung oder dem gleichen Subunternehmen.
Alle besitzen ihre eigenen Logos und Namen.
Wenn ein Termin lediglich darstellt, wann er stattfindet, müsste der Mitarbeiter immer noch nachschauen, ob der Termin für ihn gedacht ist.
Die Buchungen direkt vor dem Raum sind jedoch dafür gedacht, dass der Mitarbeiter den Raum für sich reserviert, falls er ihn jetzt gerade oder bald benötigt und dann wissen die Teilnehmer im Regelfall, dass der Termin für sie gedacht ist.
\newline
Sollte dies nicht der Fall sein, sollte der Anwender, im Voraus, an einem anderen Endgerät, welches Outlook oder Teams besitzt, einen detaillierten Termin buchen, damit alle Beteiligten wissen, wann der Termin stattfindet und wer anwesend sein wird.
Dort kann der Anwender dann auch angeben, welche Firma der Gastgeber sein soll und welche der Gast.

    %! Author = mboehme
%! Date = 23.02.2023

\section{Grundlagen}\label{sec:grundlagen}
Um die Arbeit in vollem Umfang zu verstehen, ist es wichtig, dass die Grundlagen von Webanwendungen verstanden sind.
\subsection{Webanwendungen}\label{subsec:webanwendungen}

\subsubsection{Was ist eine Webanwendung?}
Eine Webanwendung ist eine Anwendung, die über einen Browser aufgerufen werden.
Manche Webanwendungen sind auch Offline nutzbar.
\subsubsection{Vor- und Nachteile von Webanwendungen}
Eine Webanwendung ermöglicht es, dass die Anwendung von jedem Endgerät, welches über eine Internetverbindung verfügt, aus aufgerufen werden kann.
Webanwendungen können effizient, benötigen im Optimalfall wenig Speicherplatz, und können gut abgesichert werden.
Aufgrund der Tatsache, dass Browser gut standardisiert sind, können Webanwendungen schnell entwickelt werden.
\newline
\newline
Ein Nachteil von Webanwendungen ist, dass nicht alle Offline nutzbar sind.
Sie sind zudem nicht so schnell wie manche Desktopanwendungen, die auf mehr Rechenleistung zugreifen können, da sie im Regelfall Programmiersprachen nutzen, die näher an Maschinensprache sind.
\newglossaryentry{Native Sprache}{name={Native Sprache}, description={Eine Sprache, die nicht erst übersetzt werden muss, sondern direkt ausgeführt werden kann.}}
Webanwendungen besitzen heutzutage ausschließlich~\cite{JavaScript} als native Sprache.
\newglossaryentry{Frameworks}{name={Frameworks}, description={Eine Sammlung von Bibliotheken, die es ermöglichen, schneller und einfacher eine Anwendung zu entwickeln.}}
Um mehr Flexibilität und Funktionalität, ohne großen Aufwand zu ermöglichen, sind Webanwendungen auf~\gls{Frameworks}, angewiesen, welche oftmals umfangreich sind und Dependenzen besitzen, die weiterhin gepflegt werden müssen.
Dies führt oftmals dazu, dass falls eine Abhängigkeit nicht mehr gepflegt wird oder andere geupdatet werden müssen, die Anwendung eventuell nicht mehr funktioniert, falls die Abhängigkeit nicht mehr kompatibel ist.
Außerdem lässt JavaScript Laufzeitfehler zu.
%Das sind Glossareinträge. Fixen
%\subsubsection{Was ist ein Framework?}
%Ein Framework ist eine Sammlung von Bibliotheken, die es ermöglichen, schneller und einfacher eine Anwendung zu entwickeln.
%\subsubsection{Was ist eine Bibliothek?}
%Eine Bibliothek ist eine Sammlung von Funktionen, die es ermöglichen, schneller und einfacher eine Anwendung zu entwickeln.
%\subsubsection{Was ist ein Package Manager?}
%Ein Package Manager ist ein Programm, das es ermöglicht, Bibliotheken und Frameworks zu installieren und zu verwalten.
%\subsubsection{Was ist ein Package?}
%Ein Package ist eine Sammlung von Bibliotheken und Frameworks, die es ermöglichen, schneller und einfacher eine Anwendung zu entwickeln.
%\subsubsection{Was ist ein Build Tool?}
%Ein Build Tool ist ein Programm, das es ermöglicht, eine Anwendung zu kompilieren und zu verpacken.
%\subsubsection{Was ist ein Compiler?}
%Ein Compiler ist ein Programm, das es ermöglicht, eine Anwendung zu kompilieren.
%\subsubsection{Was ist ein Transpiler?}
%Ein Transpiler ist ein Programm, das es ermöglicht, eine Anwendung zu transpilieren.
%\subsubsection{Was ist eine Transpilation?}
%Eine Transpilation ist ein Prozess, bei dem eine Sprache in eine andere Sprache übersetzt wird.
%\subsubsection{Was ist eine Kompilation?}
%Eine Kompilation ist ein Prozess, bei dem eine Sprache in Maschinencode übersetzt wird.
%\subsubsection{Was ist ein Bundler?}
%Ein Bundler ist ein Programm, das es ermöglicht, mehrere Dateien zu einer einzigen Datei zusammenzufassen.
%\subsubsection{Was ist ein Linter?}
%Ein Linter ist ein Programm, das es ermöglicht, die Codequalität zu überprüfen.

%Erklären was die Microfosft Graph API ist. Nicht mehr. Aufzählung etc. weg
\subsection{Microsoft Graph API}\label{subsec:microsoft-graph-api}
Die Microsoft Graph API ist eine \gls{RESTful} web API, die es einem erlaubt auf Daten von Microsoft 365 und Office 365 zuzugreifen.
\newglossaryentry{RESTful}{
    name=RESTful,
    description={RESTful ist ein Synonym für Representational State Transfer. RESTful ist ein Architekturstil für die Entwicklung von Webdiensten.}
}

Es gibt einige APIs, die sich auf bestimmte Microsoft 365 und Office 365 Abonnements beziehen, wie beispielsweise die Microsoft Teams API\@.
Die Microsoft Graph vereint jedoch die meisten APIs, die Microsoft 365 und Office 365 betreffen, in einer einzigen API\@.
Dies ermöglicht Zugang auf Daten von Microsoft 365 und Office 365, wie beispielsweise Benutzer, Kalender, E-Mails, Dateien, etc.



    %! Author = testpcfix
%! Date = 22/02/2023

% Preamble

% Packages

% Document
\pagebreak
\section{Problemanalyse}\label{sec:problemanalyse}
\subsection{Ist-Zustand}\label{subsec:ist-zustand}
\newglossaryentry{User Journey}{
    name=User Journey,
    description={User Journey ist ein Begriff aus der User Experience Design. User Journey ist ein Weg, den ein User durchläuft, um ein Ziel zu erreichen.}
}
\newglossaryentry{UserInterface}{
    name=UserInterface,
    description={UserInterface ist ein Begriff aus der User Experience Design. UserInterface ist die grafische Oberfläche, die ein User sieht und mit der er interagiert.}
}
Die Muttergesellschaft der DOOH media GmbH und einige Schwesterunternehmen nutzen Microsoft 365 und Office 365.
%Gehört in office365 und microsoft365
Diese Produkte sind sehr umfangreich und bieten viele Funktionen.
Die heutzutage gängige und weit verbreitete Terminplanung per Outlook oder Teams ist eines dieser Funktionen.
Diese Funktion ist nicht so praktikabel wie sie es sein könnte.
Vor allem nicht, wenn mehrere Unternehmen, das gleiche Gebäude und die gleiche Organisations-E-Mail besitzen.
Daher besteht die Problematik, dass es nicht einfach zu unterscheiden ist, welcher Termin für welchen Mitarbeiter gedacht ist.
Dies könne zwar durch die Eingabe des Firmennamens, in den Textkörper des Termins, verbessert werden, jedoch müsste der Mitarbeiter dann trotzdem jeden Termin einzeln öffnen, um zu sehen, welcher Firmennamen im Textkörper steht.
\newline
\subsubsection{User Journey}
Ein anderes Kernproblem ist, dass die~\gls{User Journey}, also der Weg, den ein User durchläuft, um ein Ziel zu erreichen, meistens umständlicher ist, als notwendig.
\newline
\newline
Angenommen ein fiktiver Mitarbeiter, der den Namen Max Mustermann trägt, möchte einen Termin in einem Konferenzraum buchen, der ohnehin sich in seiner Nähe befindet.
Der Termin soll in ungefähr 15 Minuten stattfinden.
Dazu muss er zuerst Outlook oder Teams öffnen, dann den Kalender des Raumes öffnen und schauen, wann dieser Raum verfügbar ist.
Dafür muss er erstmal an einem Gerät, welches Outlook oder Teams besitzt, die Verfügbarkeit des Raumes prüfen.
Er begibt sich wieder an seinen Arbeitsplatz und findet heraus, dass der Raum in 15 Minuten verfügbar ist und bucht diesen.
Es sind schon 5 Minuten vergangen.
Da die Organisation eine Pufferzeit von 15 Minuten einplant, hat sich der nächste verfügbare Termin um 5 Minuten verschoben.
\newline
Nun möchte eine Mitarbeiterin, die den Namen Anna Musterfrau trägt, den gleichen Raum buchen.
Sie geht zufällig am Raum vorbei und möchte wissen, ob dieser Raum gleich belegt ist oder nicht.
Dafür muss sie, genauso wie Max Mustermann, zuerst Outlook oder Teams öffnen, dann den Kalender des Raumes öffnen und schauen, wann dieser Raum verfügbar ist.
Sie begibt sich wieder an ihren Arbeitsplatz und findet heraus, dass der Raum in 15 Minuten verfügbar ist.
Sie findet heraus, dass er nicht verfügbar ist, da Max Mustermann ihn bereits gebucht hat.
Für Frau Musterfrau sind nun 10 Minuten vergangen, da sie in einer anderen Abteilung des Gebäudes arbeitet.
%Warum ist das so blöd? Aus Zusammenfassung rausnehmen und hier rein
\newline
\newline
Eine schnelle Übersicht über die Verfügbarkeit des Raumes ist also nicht möglich.
Zudem setzt die heutige Terminplanung voraus, dass der Benutzer die Rechte hat, den Kalender des Raumes zu öffnen.

%\subsection{Definitionen}\label{subsec:definitionen}
%\begin{itemize}
%    \item \gls{RESTful}: RESTful ist ein Synonym für Representational State Transfer. RESTful ist ein Architekturstil für die Entwicklung von Webdiensten.
%    \item \gls{API}: API steht für Application Programming Interface. API ist eine Schnittstelle, die es einem erlaubt auf Daten zuzugreifen.
%    \item \gls{REST}: REST steht für Representational State Transfer. REST ist ein Architekturstil für die Entwicklung von Webdiensten.
%    \item \gls{HTTP}: HTTP steht für Hypertext Transfer Protocol. HTTP ist ein Protokoll, das es einem erlaubt auf Daten zuzugreifen.
%    \item \gls{JSON}: JSON steht für JavaScript Object Notation. JSON ist ein Datenformat, das es einem erlaubt Daten zu speichern und zu übertragen.
%    \item \gls{OAuth}: OAuth ist ein Protokoll, das es einem erlaubt sich mit einem Account anzumelden und somit auf Daten zuzug#
%    \item \gls{User Journey}: \gls{User Journey} ist ein Begriff aus der User Experience Design. \gls{User Journey} ist ein Weg, den ein User durchläuft, um ein Ziel zu erreichen.
%    \item \gls{UserInterface}: \gls{UserInterface} ist ein Begriff aus der User Experience Design. \gls{UserInterface} ist die grafische Oberfläche, die ein User sieht und mit der er interagiert.
%\end{itemize}
\pagebreak
\subsection{Soll-Zustand}\label{subsec:soll-zustand}
%Soll der Soll-Zustand mit einem use case beschrieben werden oder reicht es im Fließtext zu beschreiben?
\subsubsection{Anforderungen}\label{subsubsec:anforderungen}
Ein User soll am Bildschirm eines Tablets, welches vor dem Raum angebracht wird, erkennen können, ob dieser Raum belegt ist oder nicht.
Welche Termine am heutigen und nächsten Tag anstehen soll einfach ersichtlich sein.
Zudem möchte der Kunde einen Termin, am Gerät, vereinbaren können.
Der User soll also nicht mehr auf Outlook oder Teams angewiesen sein, sondern kann direkt am Bildschirm des Tablets sehen, ob der Raum belegt ist oder nicht.
Zudem soll der Raumstatus farbig dargestellt werden, sodass der User sofort erkennen kann, ob der Raum belegt ist oder nicht, sowohl am User Interface, als auch an den LED Strips des Tablets.
Die Benutzeroberfläche soll, so gestaltet sein, dass der User sofort erkennen kann, ob der Raum belegt ist oder nicht.
Des Weiteren soll das \gls{UserInterface} so gestaltet sein, dass der User auch Termine für den Raum vereinbaren kann.

\subsubsection{User Interface}\label{subsubsec:user-interface}
Das Userinterface soll also eine Übersicht über die Termine des Raumes und einen Button zum Vereinbaren eines Termins enthalten.
Es wird die Hintergrundfarbe des Containers, in dem der Raumstatus angezeigt wird, in Abhängigkeit vom Raumstatus, rot, gelb oder grün sein.
Falls ein Termin gerade stattfindet, soll dieser rot dargestellt werden, falls ein Termin innerhalb der nächsten 15 Minuten stattfindet, soll er gelb dargestellt werden und ansonsten grün.
%Neue Komponente einführen.
Sollte am heutigen Tag kein Termin stattfinden, wird der Raumstatus weiterhin grün dargestellt, jedoch wird der Text \("\)Keine weiteren Termine für Heute\("\) angezeigt.
\newline
\newline
Der erste Vorschlag für das Userinterface sieht wie folgt aus:
\newline
\newline
\begin{figure}[h]
    \par\vspace{1cm}
    \centering
    \includegraphics[width=0.8\textwidth]{Bilder/mockup}
    \caption{Mockup}
    \label{fig:Mockup}
    \par\vspace{1cm}
\end{figure}
\justifying
\newline
\newline
\subsubsection{User Flow}\label{subsubsec:user-flow}
Die hauptsächliche Benutzerinteraktion mit dem System ist das Buchen eines Termins.
Dafür wurde ein Workflow erstellt, der den Ablauf des Anwenders beschreibt und wie das System auf die Aktionen des Anwenders zu reagieren hat.
\begin{figure}[h]
\par\vspace{1cm}
\centering
\includegraphics[width=0.8\textwidth]{Bilder/Workflow Terminbuchung}
\caption{Workflow Terminbuchung}
\label{fig:Workflow Terminbuchung}
\par\vspace{1cm}
\end{figure}
\justifying
\newline
Die Rechtecke stellen die einzelnen Schritte des Anwenders dar.
Pfeile zeigen den Ablauf des Anwenders.
Entscheidungen der Software werden durch die Pfeile mit einem Pfeilspitze dargestellt.
Falls diese Entscheidung zu einer Änderung der Daten führt, wird die Folge als Parallelogramm dargestellt.
\newline
\newline
Von den Grafikdesignerinnen unseres Unternehmens und der Muttergesellschaft wurde ein zweites Design für das User Interface erstellt:

\newline
\begin{figure}[hbt!]
\par\vspace{1cm}
\centering
\includegraphics[width=0.65\textwidth]{Bilder/GrafikdesignerMockup}
\caption{GrafikdesignerMockup}
\label{fig:GrafikdesignerMockup}
\par\vspace{1cm}
\end{figure}
\justifying

\newglossaryentry{Container}{name={Container}, description={Ein Container ist ein Element, welches andere Elemente enthält.}}
Man erkennt im großen roten ~\gls{Container} den jetzigen Raumstatus.
In diesem Fall ist der Raum belegt, da er rot dargestellt wird.
Des Weiteren ist zu erkennen, von wann bis wann der Raum belegt ist, der Name des Termins und der Firmenname der Gastfirma.
\newline
\newglossaryentry{Header}{name={Header}, description={Ein Header ist ein Element, welches bei HTML Seiten, die in einem Browser angezeigt werden, oben angezeigt wird.}}
Direkt darüber sieht man den sogenannten \gls{Header}, welcher den Namen des Raumes, die Uhrzeit und das Datum anzeigt.
Außerdem ist der Gastgeber am Logo, im Header, erkennbar.
\newline
Auf der rechten Seite des Mockups, ist eine Liste an darauffolgenden Terminen zu sehen.
Sie bieten ähnliche Informationen wie der aktuelle Termin, aber zeigen nicht den Gastgeber an.
Jedoch zeigen sie an, ob der Termin am heutigen oder am nächsten Tag stattfindet und sind dementsprechend sortiert.
\newline
\newline

Dieses Design, ist jedoch in seiner Ausführung nicht praktikabel.
%Was heißt überhaupt kontrastreich
Teilweise ist der Kontrast geringer als 4.5:1, was laut~\cite{WCAG} nicht ausreichend ist.
Beispielsweise sind die dunkelgrauen Boxen mit einem fast schwarzen Hintergrund der Seite und einem fast weißen Text nicht gut differenzierbar.
Des Weiteren wurde bei der Erstellung des Designs nicht genug auf die kleine Auflösung und Displaygröße des Tablets geachtet.
Dies sind alles Eigenschaften und Variablen, die bei der Gestaltung der Benutzeroberfläche noch beachtet werden sollten.
\newline
Das Design existiert, in sechs Variationen, die sich darin unterscheiden, ob sie gerade im \("\)dunklen Design\("\) oder \("\)hellen Design\("\) und in welchem Belegungsstatus der Raum sich befindet.
\newline
\newglossaryentry{Responsive Design}{name={Responsive Design}, description={Responsive Design ist ein Begriff aus der User Experience Design. Responsive Design ist ein Design, das sich an die Größe des Bildschirms anpasst.}}

\gls{Responsive Design} wurde hier nur für das Format berücksichtigt, da die Anwendung nur für Tablets und Desktops in horizontaler Ausrichtung gedacht ist.
\newglossaryentry{DarkMode}{
    name=DarkMode,
    description={Dark Mode ist ein Begriff aus der User Experience Design. Dark Mode ist ein Modus, bei dem der Hintergrund dunkel ist und der Text hell.}
}
\newglossaryentry{LightMode}{
    name=LightMode,
    description={Light Mode ist ein Begriff aus der User Experience Design. Light Mode ist ein Modus, bei dem der Hintergrund hell ist und der Text dunkel.}
}
\newglossaryentry{Landscape}{
    name=Landscape,
    description={Landscape ist ein Begriff aus der User Experience Design. Landscape ist ein Modus, bei dem das Gerät horizontal liegt.}
}
\subsubsection{Hosting}\label{subsubsec:hosting}
\newglossaryentry{Hosting}{
    name=Hosting,
    description={Hosting ist ein Begriff aus der Informatik. Hosting ist ein Dienst, bei dem eine Firma eine Anwendung für andere bereitstellt.}
}
Die Anwendung soll lokal gehostet werden.
Durch den Verzicht auf einen Webserver werden kosten eingespart, da Updates voraussichtlich selten passieren werden.
Zudem ist es für die Sicherheit der Nutzer so viel besser, da es keinen externen Datenzugriff gibt.
\subsubsection{Authentifizierung}\label{subsubsec:authentifizierung}
Um die Anwendung zu nutzen, muss sich der Benutzer authentifizieren.
Dafür wird ein Konto benötigt, welches bei Microsoft registriert ist.
Der Nutzer muss außerdem der Anwendung erlauben, auf seine Daten zuzugreifen.
Ansonsten kann die Anwendung nicht funktionieren.
Diese Berechtigungen sollten delegiert sein.
Da die Microsoft Graph API für das Anmelden des Nutzers, Azure AD, verwendet, ist es notwendig, dass das oAuth 2.0 Protokoll ~\cite{OAuth-2.0-Simplified} verwendet wird.
Die Anwendung muss daher eine absolute, voraussagbare URL haben, die bei Azure AD registriert ist.
Dies wird nämlich die URL sein, auf die der Benutzer nach der Authentifizierung weitergeleitet wird.
\subsubsection{ISO 8601 Konformität}\label{subsubsec:iso-8601-konformitaet}
Zeiten müssen, bei jeglichen API Anfragen an die Microsoft Graph API, ISO 8601 konform sein \@.
Dies ist wichtig, da Zeitzonen des Nutzers ungleich der Zeitzonen der Server, sein können und die Microsoft Graph API nur Zeiten in UTC akzeptiert.
Zudem können Sommer- und Winterzeiten unterschiedlich sein.
Es muss also bei der Datenverarbeitung darauf geachtet werden, dass auch die Zeiten ISO 8601 konform sind.
\subsubsection{Timer}\label{subsubsec:timer}
Um zu visualisieren, wie lange der Raum noch belegt ist, soll ein Timer verwendet werden.
Dieser soll einen visuellen Countdown darstellen und einfach zu lesen sein.
\subsubsection{Fehlerbehandlung}\label{subsubsec:fehlerbehandlung}
Fehlermeldungen sollten immer so gestaltet sein, dass sie den User nicht verunsichern (~\cite{interaction-design-book1}), sondern ihn auf die Lösung des Problems hinweisen.
Auch bei der Gestaltung der Fehlermeldung sollte darauf geachtet werden, dass sie nicht zu lang ist, da der User sonst nicht mehr weiß, was er tun soll.
Bei Kiosk Anwendungen ist es besonders wichtig, dass die Fehlermeldungen kurz und prägnant sind, da der User ohnehin in einer Situation ist, in der er wahrscheinlich nicht viel Zeit hat.
Das bedeutet, dass Fehler, die dazu führen, dass die Anwendung nicht mehr funktioniert, groß und auffällig dargestellt werden sollten, damit der User sie nicht übersehen kann und versteht, dass er die Anwendung nicht mehr nutzen kann, bis das Problem behoben wurde.
Andererseits sollten Fehler, die nicht dazu führen, dass die Anwendung nicht mehr funktioniert, klein und unauffällig dargestellt werden, damit der User nicht gestört wird.
\subsubsection{LED Strips}\label{subsubsec:led-strips}
\newglossaryentry{LED Strips}{
    name=LED Strips,
    description={LED Strips sind eine Art von LED Leuchtmittel. LED Strips sind in der Regel lang und dünn und werden in Streifen angebracht.}
}
Die~\gls{LED Strips} sollen in der Farbe leuchten, die dem Raumstatus entspricht.
\subsubsection{Objektorientierte Analyse}\label{subsubsec:objektorientierte-analyse}
\newglossaryentry{Objektorientierte Analyse}{
    name=Objektorientierte Analyse,
    description={Objektorientierte Analyse ist ein Begriff aus der Softwareentwicklung. Objektorientierte Analyse ist eine Methode, um die Anforderungen an ein System zu analysieren.}
}
Anbei ein Aktivitätsdiagramm, welches den Ablauf der Anwendung beschreibt.
\par\vspace{1cm}
\begin{figure}[h]
\centering
\includegraphics[width=0.8\textwidth]{Bilder/Objektorientierte Analyse/Ressource booking activity diagramm}
\caption{Aktivitätsdiagramm}
\label{fig:Aktivitätsdiagramm}
\end{figure}
\par\vspace{1cm}
\justifying
\newline

    %! Author = Adham Aijou
%! Date = 23.02.2023
%Muss zu Soll-Zustand
\subsection{Use Case}\label{subsec:use-case}
Hierzu wurde ein Use Case erstellt, welcher die Anforderungen an das System beschreibt.
Dieser Use Case wurde in einem Tabellenformat erstellt, welches in der folgenden Tabelle dargestellt wird:
\newline
\newline
\begin{tabularx}{\textwidth}[b]{|l*{2}{|X}|}\hline
    \caption{Use Case - Terminbuchung}
    \label{tab:Terminbuchung}
    Use Case & Ressourcen- und Terminplanung in Office365 mithilfe von Microsoft Graph API\\
    \hline
    Titel & Finde freie Termine in einem Kalender für eine bestimmte Ressource\\
    \hline
    Beschreibung & Der Benutzer möchte einen Termin in einem Kalender für eine bestimmte Ressource finden.\\
    \hline
    Akteure & Benutzer, Ressource, Kalender\\
    \hline
    Vorbedingungen & Der Benutzer ist angemeldet.\\
    \hline
    Nachbedingungen & Der Benutzer hat einen Termin in einem Kalender für eine bestimmte Ressource gefunden und/oder hat einen Termin gebucht.\\
    \hline
    Normaler Ablauf & \begin{enumerate}
        \item Der Benutzer geht zu einem Tablet, welcher vor dem Raum angebracht wurde, welcher die Ressource stellvertretend repräsentieren soll.
        \item Der Benutzer sieht die verfügbaren Zeiten für die Ressource und den jetzigen Status der Ressource.
        \item Der Benutzer wählt einen Termin aus.
        \item Der Benutzer bucht den Termin.
        \item Der Benutzer erhält eine Bestätigung oder eine Fehlermeldung.
    \end{enumerate}\\
    \hline
\end{tabularx}
\newline
\newline
%Dazu gibt es auch ein Diagramm, welches den Ablauf des Use Cases beschreibt:
%\newline
%\newline
%\begin{figure}[h]
%\par\vspace{1cm}
%\centering
%\includegraphics[width=\textwidth]{Bilder/Objektorientiertes Design/Use Case diagram ressource booking}
%\end{figure}
%\justifying
\newline
\newline

    %! Author = Adham Aijou
%! Date = 23.02.2023

\subsection{Systemschnittstellen}\label{subsec:systemschnittstellen}
%first column is too wide, make everything fit into one A4 page
\begin{tabularx}{\textwidth\footnotesize}{|l*{2}{|X}|} \hline
%    \multicolumn{2}{|c|}{\textbf{Microsoft Graph API}}\\
    \caption{Termin buchen}
    \label{tab:TerminBuchen}
 Kurzbeschreibung: & Im UI kann der User sich mit der Ressource einloggen, die für die Zukunft als Repräsentant für die reelle Ressource (z.B einen Raum) gelten soll.
    Dann gilt dieser User immer als Teilnehmer der Buchungen und kann in seinen verfügbaren Zeiten gebucht werden.
    Der User kann zudem andere Mitarbeiter hinzufügen, um sie einerseits über das Meeting zu informieren und andererseits, um automatisch überprüfen zu können, wann alle am besten Zeit haben.
    Darauf basierend werden dem User verfügbare timeslots zur Buchung angezeigt.
    Auch die Dauer des Termins soll einstellbar sein.
    Es gibt zudem einen optionalen Betreff für den Termin.
    Der gebuchte Termin reflektiert sich dann im UI in der Liste der für den Tag schon gebuchten Termine, für die Ressource.
    Zudem soll der Nutzer sehen können ob innerhalb der nächsten 30 Minuten ein Termin für die Ressource ansteht oder ob einer schon am Passieren ist, anhand einer farblichen UI .

    REST API Anfragen werden dann an die Microsoft Graph API geschickt, um diese Wünsche zu kommunizieren.\\
    \hline
    Akteure: & Mitarbeiter\\
    \hline
    Häufigkeit: & Variablel pro Ressource, Firma und Anzahl Ressourcen\\
    \hline
    Komplexität: & Mittlere bis hohe Komplexität\\
    \hline
    Vorbedingungen: & \begin{enumerate}
                          \item User muss eingeloggt sein (Nutzerdaten können nicht gesehen oder eingespeichert werden) und Zugriffsrechte akzeptieren.
                          \item Optionale Teilnehmer
                          \item Termindauer
                          \item Verfügbare Termine
                          \item Terminbetreff
                          \item Zeitzone
                          \item Für die Microsoft Graph API kommt noch der Access Token / Client ID von unserer Azure App zur Authentifizierung dazu
    \end{enumerate}\\
    \hline
    Ausgaben: & \begin{enumerate}
                    \item Buchung erfolgreich oder nicht
                    \item Neue verfügbare Termine
                    \item Der neue Termin wird in der Liste der gebuchten Termine angezeigt
                    \item Falls die Buchung nicht geklappt haben sollte, wird der User darüber informiert
    \end{enumerate}\\
    \hline
    Use Case: &~\ref{tab:Terminbuchung}\\
    \hline
Schnittstellen: & \begin{enumerate}
                        \item Microsoft Graph API
                        \item User Interface
                        \item Backend
                      \end{enumerate}\\
    \hline
    Aufgerufene Aktionen: & Notwendige: \begin{enumerate}
                                        \item Termin buchen
                                        \item Termin finden
                                            \end{enumerate}
            \linebreak Optionale: \begin{enumerate}
                                      \item Teilnehmer hinzufügen
                                      \item Terminbetreff hinzufügen/ändern
                                        \item Termindauer ändern
                                      \end{enumerate}\\
    \hline
\end{tabularx}
\pagebreak
    %! Author = mboehme
%! Date = 23.02.2023

% Preamble
\section{Abgrenzung zu ähnlichen Produkten}
\subsection{Microsoft Teams}
Microsoft Teams ermöglicht Terminbuchungen genauso, erlaubt jedoch zu detaillierte Eingriffe in das Konto, welcher für den Raum vorgesehen ist.
Einstellungen, die nicht verändert werden sollten, können dann erreicht werden.
\newline
\subsection{Outlook}
Outlook ist ein E-Mail-Programm, welches ebenfalls Terminbuchungen ermöglicht.
Outlook ist jedoch nicht für die Terminbuchung in Konferenzräumen gedacht und kann je nach E-Mail-Konto-Typ nicht in vollem Umfang genutzt werden und ist zudem nicht immer vollständig synchronisiert.
Auch hier können Einstellungen, die nicht verändert werden sollten, erreicht werden und Kalender eingesehen werden, die nur für Administratoren gedacht sind.
\newline
\subsection{Microsoft Bookings}
Microsoft Bookings ist eine weitere Applikation, die Terminbuchungen ermöglicht.
Sie wird hauptsächlich für Terminbuchungen in beispielsweise Friseursalons, Fitnessstudios oder ähnlichen Geschäften genutzt.
Dies ist jedoch der falsche Anwendungsfall für die Terminbuchung in Konferenzräumen.
Auch die Einschränkungen und Freiheiten, die diese Applikation bietet, sind hier nicht passend.
\newline
\subsection{Microsoft Power Automate}
Microsoft Power Automate ist eine Applikation, die es ermöglicht, Workflows zu erstellen.
Diese Workflows können dann automatisiert ausgeführt werden.
Diese kann auch ähnliche Funktionen wie die Applikation, die in dieser Arbeit entwickelt wird, bieten.
Jedoch ist die Freigabe dieser Applikationen nur für innerhalb der Organisationen vorgesehen und nicht für Kunden.
\pagebreak
\subsection{Spezifische Gründe für die Entwicklung einer eigenen Applikation}
Wenn ein Kunde bei der DOOH media GmbH einkauft, kauft er im Regelfall ein einigermaßen volles Paket.
Dieses Paket inkludiert die Möglichkeit die Geräte in einen sogenannten Kiosk-Modus zu versetzen.
Dafür haben wir eine eigene Applikation, die Oxygen Player App, entwickelt.
Zudem haben wir eine \(")\)OMS\(")\)-Lösung (Oxygen Media Server), die es ermöglicht, die Inhalte auf den Geräten zu verwalten.
Falls der Kunde also zu bestimmten Zeiten eigene Inhalte oder Inhalte von diversen Anbietern, auf den Geräten, zu bestimmten Uhrzeiten, darstellen möchte, kann er dies über die \(")\)OMS\(")\)-Lösung tun.
Diese Inhalte werden dann auf den Geräten abgespielt.
\newline
Die Idee ist hier also, dass der Kunde nicht an die Terminbuchungs-Webapplikation gebunden ist, sondern diese Applikation nur als zusätzliche Funktion anbietet.
Viele der Softwarelösungen, die schon für genau diesen Anwendungsfall existieren, beinhalten Bindungen an andere Softwarelösungen, die der Kunde nicht benötigt.
Sie machen es dem Kunden schwer, die Softwarelösung zu nutzen, da er sich mit anderen Softwarelösungen auseinandersetzen muss, die er nicht benötigt und gegebenfalls nicht in einem Kiosk-Modus laufen können oder parallel zu unserer restlichen Software, die der Kunde haben möchte.
Außerdem erforden einige dieser Softwarelösungen eine Installation auf dem Gerät oder erzwingen Konten-Registrierungen, die der Kunde nicht benötigt.
\newline
Die Applikation, die in dieser Arbeit entwickelt wird, soll es hingegen ermöglichen, dass diese Softwarelösung ohne weitere Softwarelösungen genutzt werden kann, aber auch ohne Probleme mit anderen Softwarelösungen genutzt werden kann, solange die anderen Softwarelösungen, von sich aus, keine Konflikte mit der Terminbuchungs-Webapplikation verursachen.
Alles wird lokal auf dem Gerät abgespielt und es wird keine Installation auf dem Gerät benötigt.
Es wird keine Registrierung eines Kontos benötigt und niemand kann die Daten von außerhalb, außer natürlich Microsoft, einsehen, da wirklich alles lokal gehostet wird.
Die Inhalte werden dynamisch auf dem Gerät geladen und verarbeitet.

    %! Author = Adham Aijou
%! Date = 22/02/2023

% Preamble
\pagebreak
\section{Vorgehensweise}\label{sec:vorgehensweise}

\subsection{Prototyp}\label{subsec:prototyp}
Erst wurde ein Prototyp mithilfe von vuejs entwickelt, basierend auf folgendem Mockup, welcher erstellt wurde bevor die Grafikdesigner ein Konzept erstellt hatten:
%Das ist Vorschlag 1. Muss in Soll-Zustand geändert werden
\par\vspace{1cm}
    \centering
    \includegraphics[width=0.8\textwidth]{Bilder/mockup}
    \caption{Mockup}
    \label{fig:Mockup}
\par\vspace{1cm}
\raggedright
    %! Author = mboehme
%! Date = 23.02.2023

\subsection{Technische Umsetzung}
\subsubsection{Grobe Planung}
Das Ganze wird als eine Azure App deployt.
Diese lässt localhost Verbindungen zu und gibt einem die Möglichkeit das ganze als Single Page Application zu entwickeln, mit einem lokalen cookie cache für den eingeloggten Account.
Somit ist keine Individualisierung notwendig.
\newline
Es wird, lokal, mithilfe von Dory-node.js gehostet.
Diese lässt localhost Verbindungen zu und gibt einem die Möglichkeit die Web-Applikation wird als Single-Page-Application, mit einem lokalen cookie cache für den eingeloggten Account, entwickelt.
Somit ist keine Individualisierung notwendig.
Zudem kann dann mit dem Phillips Display die Web-Applikation aufgerufen werden und als eine Art Model eines Model-View-Controllers verwendet werden.
\newline
Der Player braucht auf der OMS den Content-Typen \("\)Web\("\) mit der URL: \("\)http://localhost:3000/content\("\).
Diese Seite wird dann lokal vom Player aufgerufen, worauf der Dory-nodejs Server antwortet und die tatsächliche Seite anbietet, die lokal auf dem Display liegt.
So ist die URL durchgehend, auf allen Geräten, identisch, damit die OAuth 2.0 URI Bedingungen ~\cite{OAuth-2.0-Simplified} erfüllt sind.
Durch den Verzicht auf einen Webserver werden kosten eingespart, da Updates eher selten passieren sollten.
\newline
Zudem ist es für die Sicherheit der Nutzer so viel besser, da wir deshalb keinen Zugriff auf ihre Daten haben.
Zeiten müssen ISO 8601 Konform sein bei jeglichen API Anfragen an die Microsoft Graph API\@.
\newline
\pagebreak
\subsubsection{Benutzte Hardware und Software}
Hardware:
\begin{itemize}
    \item Philips 10BDL4551T/00 (Firmware: FB01.13, SICP Version: v.2.05)
\end{itemize}
\newline
Die Hardware ist ein Philips~\cite{10BDL4551T/00} Display.
Dieses Display ist ein 10 Zoll (ca. 25 cm) Touchscreen.
Es hat eine Auflösung von 1280x800 Pixeln und benutzt Android.
Dieses Display ist für Digital Signage gedacht und kann so auch als solches verwendet werden.
Die eingebauten RGB LED Strips können per SICP angesteuert werden.
\newline
\newline
Software:
\begin{itemize}
    \item Vue.js 3.x
    \item Microsoft Graph API per NPM Package
    \item Dory-Node.js
    \item Azure App
    \item Babel
    \item Webpack
    \item Node.js
    \item NPM
    \item Git
    \item EsLint
    \item IntelliJ IDEA
\end{itemize}
\newline
Vue.js ist ein JavaScript Framework und transpiliert die Vue.js Dateien in JavaScript, damit sie von jedem Browser ausgeführt werden können.
Es ist darauf spezialisiert, Single Page Applikationen zu entwickeln.
\newline
\newline
Dory-Node.js ist ein Node.js Server, der es ermöglicht, eine Web-Applikation lokal zu hosten.
Dies wird benötigt, um einerseits die Raumbuchungsseite lokal anzubieten, damit die absoluten Redirect-URI Bedingungen von Microsoft und oAuth 2.0 erfüllt werden können und andererseits, um die Anwendung auf dem Display zu hosten und als Schnittstelle zwischen der Seite und dem den LED-Strips des Displays zu agieren.
\newline
\newline
Die Microsoft Graph API wird per NPM Package verwendet, um die Anfragen an die Microsoft Graph API zu vereinfachen und den User einzuloggen.
\newline
\newline
Babel ist ein JavaScript Compiler, der es ermöglicht, moderne JavaScript Features zu verwenden, die von älteren Browsern nicht unterstützt werden, um so die Kompatibilität zu erhöhen.
\newline
\newline
Webpack ist ein JavaScript Bundler, der es ermöglicht, mehrere JavaScript Dateien zu einer einzigen zusammenzufassen, um so die Ladezeiten zu verkürzen.
\newline
\newline
Node.js ist ein JavaScript Runtime, der es ermöglicht, JavaScript Code auszuführen, ohne einen Browser zu benötigen.
Es ist das sogenannte Backend dieser Anwendung.
Dieser wird per Dory-Node.js gehostet.
\newline
\newline
NPM ist ein Package Manager, der es ermöglicht, JavaScript Packages zu installieren und zu verwalten.
\newline
\newline
Git ist ein Versionskontrollsystem, welches es ermöglicht, Änderungen an Dateien zu verfolgen und zu verwalten.
Dies wurde aber in Zusammenhang mit GitLab verwendet, da dies der derzeitige, etablierte Standard der DooH media GmbH ist.
\newline
\newline
EsLint ist ein Linter, der es ermöglicht, JavaScript Code zu analysieren und zu formatieren.
Da Testfälle bei JavaScript Projekten nicht immer vollständig möglich sind, ist es wichtig, dass der Code einheitlich ist und keine Fehler enthält.
Deshalb achtet EsLint auf die Einhaltung von Regeln, damit erst gar keine Fehler entstehen, die der JavaScript Compiler nicht erkennen kann.
Beispielsweise wird überprüft, ob Variablen deklariert wurden, bevor sie verwendet werden und anders herum.
\newline
\newline
IntelliJ IDEA ist eine IDE, die es unter anderem ermöglicht, JavaScript Code zu schreiben und zu debuggen.
\newline
\newline

\subsubsection{Weiterentwicklung der groben Planung}
Es wurde eine Azure App erstellt, die die Authentifizierung der Anwendung und des Users übernimmt.
\newline
Dort wird der User eingeloggt und die Anwendung leitet ihn auf die Seite weiter, die er vorher besucht hat, welche in diesem Fall, die Raumbuchungsseite ist.
\newline
Es wurde jede Woche Rücksprache mit dem Kunden gehalten, um die Anforderungen zu besprechen und zu erfüllen.
Aber auch intern wurde Rücksprache gehalten, was denn sinnvoll ist und was nicht.
\newline
So wurde jede Woche die Anwendung weiterentwickelt und verbessert.
Teilweise wurden auch neue Features hinzugefügt, die nicht im Pflichtenheft standen, aber sinnvoll waren.
Manche Features wurden auch wieder entfernt, da sie nicht sinnvoll waren oder von anderen Features abgedeckt wurden.
\newline
%Ablauf der Entwicklung
\newline
\subsubsection{Ablauf der Entwicklung}
Nachdem der erste Prototyp fertig war und Rücksprache gehalten wurde, wurde angefangen die Anwendung zu entwickeln.
Design und Funktionalität wurden dabei partiell parallel entwickelt, wobei die Funktionalität immer Priorität hatte.
Über 95 Commits wurden die Änderungen an der Anwendung festgehalten.
\newline
Es wurde ein Testgerät benötigt, um die Anwendung zu testen.
Ein Phillips 10BDL4551T/00 Display wurde dafür an die Wand gehängt und mit dem Internet verbunden.
Zudem wurde es so eingerichtet, wie es auch beim Kunden laufen soll.
\newline
Es wurde täglich aktiv genutzt, um so Fehler zu finden und Feedback zu geben.
Das Feedback wurde dann, falls sinnvoll, in die Anwendung eingearbeitet.
Solche praktischen Tests sind sehr wichtig, um Intuitivität und Benutzerfreundlichkeit zu gewährleisten.
Es war einerseits hilfreich, um vorgesehene Abläufe zu testen, andererseits aber auch um Fehler zu finden, die nicht vorgesehen waren, indem Monkey Testing~\cite{Monkey-Testing-Book1} betrieben wird.
\newline

Mithilfe der oDatav4 (~\cite{oData}) API wurden die Daten aus der jeweiligen Microsoft Datenbank im Voraus gefiltert, um so nicht notwendige Daten zu vermeiden und Performance drastisch zu erhöhen, als auch Datenvolumen zu sparen.
Die Daten wurden dabei in JSON Format zurückgegeben.
Die oData v4 API verhält sich dabei wie eine SQL Datenbank, wobei die Daten in Tabellen gespeichert sind.
Es wurden die Klauseln \("\$\)filter\("\) und \("\$\)top\("\) verwendet, um nur Termine für den aktuellen Tag und nächsten Tag zu erhalten und dies einzuschränken auf die top 300 Termine, falls jemand versucht das System zu überlasten.
Eine Sortierung ist nicht notwendig, da die Termine bereits nach Startzeit sortiert sind.
Hier sieht man die oData v4 API Anfrage:
\newline

%add javascript code formatting
\begin{lstlisting}[language=javascript,label={lst:JavaScript oData v4 API Anfrage}]
     let url =  "https://graph.microsoft.com/v1.0/me/findMeetingTimes/?$filter=start/dateTime" +  "ge"  + "${todayDate} and end/dateTime le ${tomorrowDate}&$top=300";
\end{lstlisting}
\newline
\newline
\subsubsection{Timer}
Der Timer, welcher die übrig bleibende Zeit bis zum nächsten Termin anzeigt funktioniert, wie folgt:
\newline
\newline
Es wird die Zeit bis zum Ende den jetzigen Termin berechnet, ausgehend von der Zeit, die zum Anfang des jetzigen Termins vergangen ist, und in Millisekunden umgerechnet.
Dann wird eine sich drehende Animation erstellt, die für diese Dauer abläuft.
\newline
Da die Dauer angepasst werden kann, muss bei Datenänderungen die Animation neu berechnet werden.
Damit dies nicht ohne Grund passiert, wird geprüft, ob die Dauer sich geändert hat.
Die Daten dafür werden im Local Storage gespeichert.
Falls eine Änderung stattgefunden hat, wird die Animation neu berechnet.
Bei der Änderung werden, damit die Animation visuell nicht von vorne anfängt, die Animationsdauer und die vergangene Zeit als neue Animationsdauer aufaddiert und die vergangene Zeit als negative Animationsverzögerung gesetzt, damit die Animation visuell betrachtet dort ist, wo sie relativ zur neuen Dauer sein sollte.
\newline
\newline
Die Formel lautet, wie folgt:
\newline
\newline
\begin{equation}
\begin{aligned}
    \text{Vergangene Zeit} &= \text{Jetzt} - \text{Start} \\
\text{Animationsdauer} &= \text{Ende} - \text{Start} \\
    \text {Animationsverzögerung} &= \text{-Vergangene Zeit} \\
\end{aligned}
\end{equation}
\newline
\newline
Man kann die Formel verkürzen und die Animationsdauer und -verzögerung direkt berechnen, aber es ist verständlicher, falls jemand in der Zukunft dies nachvollziehen möchte.
\newline
\newline
Hier sieht man den zugehörigen JavaScript Code:
\newline
\newline
\begin{lstlisting}[language=javascript,label={lst:JavaScript Timer}]
let currentEventBeginningTime = localStorage.getItem('currentEventBeginningTime');
let timePassed = (new Date() - new Date(currentEventBeginningTime + 'Z')) / 1000;
let animationDuration = ((new Date(currentEventEndTime + 'Z') - new Date(currentEventBeginningTime + 'Z')) / 1000);
firstHandSpan.style.animationDuration = animationDuration  + 's';
secondHandSpan.style.animationDuration = animationDuration + 's';
firstHandSpan.style.animationDelay = -timePassed + 's';
secondHandSpan.style.animationDelay = -timePassed  + 's';
\end{lstlisting}
\newline
\newline
Um die Animation während ihrer Laufzeit zu ändern, wird ein sogenannter Reflow ~\footcite{JavaScriptAnimationsReflow} erzwungen, indem die offsetWidth Eigenschaft abgefragt, die Animationsdauer und -verzögerung neu gesetzt und der Animationsname erst entfernt und dann wieder hinzugefügt wird.
\newline
\newline
\subsubsection{Persistente Datenspeicherung}
Um die Daten der Anwendung zu speichern, wurde für kleinere Datensätze der Local Storage verwendet.
Dieser ist in jedem Browser verfügbar und darf bis zu 5 MB groß sein.
\newline
\newline
Für größere Datensätze wurde IndexedDB verwendet.
Dies ist eine NoSQL Datenbank, die in jedem Browser verfügbar ist.
Die maximale Größe bei Chrome beträgt 80\(\%\) des verfügbaren Speichers.
Da es im Internet viele, sich widersprechende Angaben zu dieser Größe gibt, wurde diese manuell, mit folgendem Code, in der Konsole des Browsers, getestet:
\newline
\newline
\begin{lstlisting}[language=javascript,label={lst:JavaScript IndexedDB Speichergröße}]
    if (navigator.storage && navigator.storage.estimate) {
    const quota = await navigator.storage.estimate();
    // quota.usage -> Number of bytes used.
    // quota.quota -> Maximum number of bytes available.
    const percentageUsed = (quota.usage / quota.quota) * 100;
    console.log("You've used ${percentageUsed}% of the available storage.");
    const remaining = quota.quota - quota.usage;
    console.log("You can write up to ${remaining} more bytes.");
    }
\end{lstlisting}
\newline
\newline
Die Festplatte, auf der dies getestet wurde, hatte ca. 828 GB freien Speicher.
Das Ergebnis lautete wie folgt:
\newline
\newline
\begin{lstlisting}[language=javascript,label={lst:JavaScript IndexedDB Speichergröße Ergebnis}]
    You've used 0.000002393592594674544% of the available storage.
    You can write up to 599726105785 more bytes.
\end{lstlisting}
\newline
\newline
%equation
\begin{equation}
    \begin{aligned}
       \text{verbleibender Speicher} &= \text{599726105785 Bytes}\\
       \newline
       \text{599726105785 Bytes} \text{ * 10}\textsuperscript{-9} &= \text{599,726105785 Gigabytes}\\
        \newline
        \newline
    \end{aligned}
\end{equation}
\newline
\newline
Das sind etwas über 70 \%\) der verfügbaren Speicherkapazität.
Hier ist davon auszugehen, dass die restlichen 10 \%\) für andere Daten schon verwendet werden.
Zudem ist zu betrachten, dass die \"\).\"\) Schreibweise für Dezimalzahlen in JavaScript verwendet wurde, während wiederum die \"\),\"\) Schreibweise für die mathematische Gleichung verwendet wurde.
Dies hat den Hintergrund, dass die \"\),\"\) Schreibweise in JavaScript nicht grundsätzlich unterstützt wird, beziehungsweise, die Konvention beim Programmieren verlangt, dass man auf Englisch programmiert und im Englischen die Dezimalzahlen mit einem Punkt geschrieben werden.
\newline
\newline
Die Datenbank befindet sich aufgrund ihrer geringen Komplexität in der 5. Normalform.
Der Schlüssel besteht aus dem Namen der Firma, zu der, das Logo gehört und gibt das nicht Primärattribut zurück, welches das Bild, in Base64, enthält.
Schlüssel müssen eindeutig sein.
IndexedDB unterstützt die Eigenschaft \"\).unique\"\), welche dafür sorgt, dass der Schlüssel eindeutig ist.
Jedoch gehen wir auch manuell noch einmal sicher, dass der Schlüssel eindeutig ist, also dass sich dieser, noch nicht in der Datenbank befindet.
Eine weitere Aufteilung würde zu Informationsverlust führen, da der Firmenname eindeutig ist.
\newline
\newline
Zwar gibt es die Möglichkeit die Bilder für Firmen, zu updaten, welcher der User auch so wahrnimmt, jedoch wird der Schlüssel des alten Bildes entfernt und ein neuer Schlüssel mit dem neuen Bild hinzugefügt.
Dies ist notwendig, da der Schlüssel eindeutig sein muss.
Somit löschen wir den alten Eintrag und fügen einen neuen hinzu.
Unabhängig von der Interaktion mit der Datenbank, wird sichergestellt, dass sinnfreie Anfragen an die Datenbank nicht gesendet werden.
Falls der User beispielsweise ein Bild hochlädt, welches bereits in der Datenbank, für genau den gleichen Schlüssel, vorhanden ist, wird die Anfrage an die Datenbank nicht gesendet oder falls der User versucht ein Bild aus der Datenbank zu entfernen, welches gar nicht existiert.
\newline
\newline

    %! Author = mboehme
%! Date = 01.03.2023

\newpage
\section{Ergebnis}\label{sec:ergebnis}
Abgleich der Ergebnisse mit den Soll-Anforderungen.
\subsection{Visuelle Darstellung}\label{subsec:visuelle-darstellung}
%include two images
\par\vspace{1cm}
\begin{figure}[h]
    \centering
    \includegraphics[width=0.8\textwidth]{Bilder/Ergebnis}
    \caption{Ergebnis mit nächstem anstehenden Termin}
    \label{fig:Ergebnis mit nächstem anstehenden Termin}
\par\vspace{1cm}
\end{figure}
\justifying
\par\vspace{1cm}
\begin{figure}[h]
    \centering
    \includegraphics[width=0.8\textwidth]{Bilder/Ergebnis_LaufenderTermin}
    \caption{Ergebnis mit laufendem Termin}
    \label{fig:Ergebnis mit laufendem Termin}
\par\vspace{1cm}
\end{figure}
\justifying
\newline
Oben links im Bild, ist das Logo des Gastgebers des nächsten, beziehungsweise jetzigen, Termins zu sehen.
In diesem Fall ist dies das Logo der DOOH media GmbH\@.
Solch ein Logo kann dargestellt werden, indem beim Erstellen des Termins, außerhalb des Tablets, bei Outlook beispielsweise, ein Bild an den Termin angehangen wird, welches im Dateinamen \("\)Termin\_Logo\("\) enthält.
\newline
Die anderen Logos sind alle Logos vom Gast des Termins.
Sie werden angezeigt, indem der Firmenname im Textkörper des Termins vorkommt.
Um diese Logos initial den Firmennamen zuzuordnen, muss ein spezieller Termin erstellt wird, der nur für die Logos gedacht ist und eine einzigarte ID, sowie einen Befehl enthält, die dann das Bild, inklusive Firmennamen, in einer lokalen Datenbank abspeichert.
Diese Logos können hinzugefügt, gelöscht oder aktualisiert werden.
\newline
Die Uhrzeit wird immer in der Zeitzone des Tablets angezeigt.
\newline
\newline
%Das gehört in sollzustand.
Hier sieht man das Menü, in welchem ein Termin gebucht werden kann, welches durch das Drücken des Plus-Symbols aufgerufen wird:
\begin{figure}[h]
\par\vspace{1cm}
    \centering
    \includegraphics[width=0.8\textwidth]{Bilder/Ergebnis_TerminErstellen_Menue}
    \caption{Terminbuchungsmenü}
    \label{fig:Menue}
\par\vspace{1cm}
\end{figure}
\justifying

Die \("\)Jetzt\("\) Option wird im Terminbuchungsmenü nicht angezeigt, weil bereits ein Termin stattfindet.
Falls ein kein Termin stattfindet, wird eine \("\)Jetzt\("\) Option angezeigt.
Wenn das Ende des \("\)Jetzt\("\) Termins innerhalb von 15 Minuten vom nächsten anstehenden Termin beginnt, wird der \("\)Jetzt\("\) Termin rot markiert, sodass der Nutzer weiß, dass er den Termin nicht buchen kann, weil er zu lange dauert.
Diese Option ist dann auch deaktiviert.
So wird dem Nutzer übermittelt, dass die Option generell schon verfügbar ist, aber nur falls die Dauer verkürzt wird.
Es werden alle möglichen Termine, innerhalb der Arbeitszeiten, für den Jetzigen und nächsten Tag angezeigt.
Falls z.B. der nächste Tag ein Feiertag ist, werden für den nächsten Tag keine Termine angezeigt.
Auf die anderen Optionen, wie z.B. Pufferzeiten, hat ein Anwender hier wenig Einfluss, da sie von den Einstellungen des Kalenders, des jeweiligen Nutzers, abhängen und es somit in der Verantwortung des Administrators liegt, diese zu ändern.
\newline
\newline
Hier die normale Ansicht nochmal, im hellen Design:
\par\vspace{1cm}
\begin{figure}[h]
    \centering
    \includegraphics[width=0.8\textwidth]{Bilder/Ergebnis_lightMode}
    \caption{Ergebnis im hellen Design}
    \label{fig:Ergebnis im hellen Design}
\par\vspace{1cm}
\end{figure}
\justifying
\newline
\newline
\pagebreak
%Eigenes Kapitel

\subsection{Ausblick}\label{subsec:ausblick}
\subsubsection{kurzfristig}\label{subsubsec:kurzfristig}
Die Software wird vom Kunden eingesetzt.
Da dies ein Pilotprojekt ist, wird die Software nur von einem Kunden genutzt.
Dafür wurden acht Tablets angeschafft, die an den Räumen des Kunden installiert werden.

\subsubsection{mittelfristig}\label{subsubsec:mittelfristig}
Der Kunde wird die Anwendung erstmal benutzen müssen.
Rückmeldungen werden gesammelt und kritische Fehler werden behoben.
Trotz ausgiebigen Tests, kann es immer noch zu Fehlern kommen, die erst im Einsatz auffallen.
\subsubsection{langfristig}\label{subsubsec:langfristig}
Die Schnittstelle wurde nicht vollständig genutzt, da die Anwendung nicht alle Funktionen benötigt, aber sie bietet für die Zukunft viele Erweiterungsmöglichkeiten.
Mit der gesammelten Erfahrung und den gewonnenen Erkenntnissen kann die Anwendung in Zukunft erweitert werden.
Zudem ist die API neu und wird ständig weiterentwickelt~\cite{microsoft-graph-api-version}.
Insbesondere mit den jüngsten Entwicklungen von künstlichen Intelligenzen, wie z.B. der Spracherkennung, könnte die Anwendung erweitert werden.

\subsection{Fazit}\label{subsec:fazit}
%Für die Anwendung wurde die Schnittstelle genutzt, um Termine zu erstellen, zu löschen und zu aktualisieren.
%Außerdem wurde die Schnittstelle genutzt, um die Logos der anderen Termine zu erhalten.
%
%Für den Anwendungsfall des Kunden ist die API ausreichend, da die Anwendung nur Termine erstellen, löschen und aktualisieren können muss.
%Alles Weitere wird von der Webapplikation abgedeckt.
%Der einzige Nachteil ist, dass Azure AD, die Authentifizierung, für Ressourcenkonten nicht unterstützt, die beispielsweise, in Teams, genutzt werden.
%Daher muss der Anwender sich mit einem Microsoft-Konto anmelden, welcher die Ressource repräsentieren soll, aber nicht ein Ressourcenkonto ist.
%Dies kann zu Verwirrung führen, da die Terminologie impliziert, man könne sich mit einem Ressourcenkonto anmelden.
%\newline
%\newline
%Die API war die richtige Wahl für diesen Kundenauftrag.
%Ressourcen- und Terminplanung wurde mit der Anwendung vereinfacht.
%Ein Anwender braucht nur noch drei Klicks, um einen Termin zu erstellen und muss nicht mehr zwischen verschiedenen Kalendern hin und her wechseln.
%%Aua, Fixen, umformen
%Zudem benötigt ein Anwender lediglich einen Blick, um zu sehen, wann die Ressource, die durch die Software repräsentiert wird, frei ist und wann der Anwender gegebenenfalls einen Termin besitzt (Anhand der Logos).
%\newline
%\newline
%Die Anwendung kann auch für menschliche Ressourcen genutzt werden, da die Anwendung keine Unterscheidung zwischen menschlichen und nicht-menschlichen Ressourcen macht.
%Solche Entscheidungen obliegen dem Anwender.
\newline
\newline
Von den Zielen wurden alle erreicht (Siehe~\ref{subsec:ziele-der-arbeit}).
Die Implementierung, mithilfe der Microsoft Graph API, wurde als sinnvoll eingeschätzt (Siehe~\ref{sec:abgrenzung-zu-ahnlichen-produkten}) und wurde dementsprechend auch implementiert~ref{sec:technische-umsetzung}.
Diese ermöglicht synchronisierte Ressourcen- und Terminplanung mithilfe der selbst entwickelten Anwendung.
Die Anwendung erfüllte mithilfe der Microsoft Graph API alle Anforderungen des Kunden(Siehe ~\ref{subsec:soll-zustand}).
Mithilfe der minimalen Benutzerführung, die die Anwendung bietet, kann der Anwender schnell und einfach Termine erstellen, löschen und aktualisieren.
Die Leistung der Anwendung wurde mithilfe des RAIL-Modells bewertet(Siehe~\ref{subsec:performanztests}) und erfüllte alle Anforderungen im vollen Umfang.
Die Software ist durch den Einsatz von Babel rückwärtskompatibel und kann somit auch von älteren Browsern genutzt werden.
Aufgrund der sauberen Auftrennung von Funktionen und Komponenten, ist die Anwendung gut wartbar und erweiterbar.
Die Anwendung ist auch für die Zukunft gut gerüstet, da die Microsoft Graph API ständig weiterentwickelt wird.
Abschließend ist die Webapplikation kostengünstig, da sie lokal gehostet werden kann(Siehe~\ref{subsubsec:hosting})

%\pagebreak
\newline
%Abschließend folgt noch ein Bild des Gerätes mit der Anwendung, welches im Kundenbetrieb genutzt wird.
\newline
%Überleg dir was Anderes. Das ist hässlich.
%\par\vspace{1cm}
%\begin{figure}[h]
%    \centering
%    \includegraphics[width=0.5\textwidth]{Bilder/FertigesProdukt}
%    \caption{Fertiges Produkt - Gerät mit der Anwendung}
%    \label{fig:fertiges-produkt}
%\newline
%\newline
%    \end{figure}
\justifying
\newpage
\section{Literaturverzeichnis}\label{sec:literaturverzeichnis}
%    cite without square brackets
    ~\nocite{*}
    \bibliographystyle{alpha}
    \printbibliography
    \newpage
    \printglossaries
    \newpage
    %! Author = mboehme
%! Date = 20.03.2023

\section{Anhang}\label{sec:anhang}
\newline
\newline
\subsection{Objektorientertes Design}\label{subsec:objektorientiertes-design}
\newline
\newline
\par\vspace{1cm}
\centering
\includegraphics[width=0.5\textwidth]{Bilder/Objektorientiertes Design/Sequence diagram for ressource booking (3)}
\caption{Sequenzdiagramm für die Anwendung}
\label{fig:sequence-diagramm}
\newline
\newline
\par\vspace{1cm}
\centering
\includegraphics[width=0.5\textwidth]{Bilder/Objektorientiertes Design/Use Case diagram ressource booking}
\caption{Use-Case-Diagramm für die Anwendung}
\label{fig:use-case-diagramm}
\newline
\newline
\newpage
\subsection{Objektorientierte Analyse}\label{subsec:objektorientierte-analyse}
\par\vspace{1cm}
\centering
\includegraphics[width=0.5\textwidth]{Bilder/Objektorientierte Analyse/Ressource booking activity diagramm}
\caption{Aktivitätsdiagramm für die Anwendung}
\label{fig:aktivitaetsdiagramm}
\newline
\newline
\par\vspace{1cm}
\centering
\includegraphics[width=0.5\textwidth]{Bilder/Raumanzeige_grob.jpeg}
\caption{Raumanzeige Mockup vom ersten Meeting}
\label{fig:raumanzeige-mockup}
\newline
\newline
\newpage

\par\vspace{1cm}
\centering
\includepdf[pages=1,pagecommand={},scale=0.8,pagecommand={\subsection{Spezifikationsblatt}\label{subsec:spezifikationsblatt}}, width=\textwidth]{PDFs/Spezifikationsblatt - Raumbuchungsanzeige - Office365 room booking - censored details.pdf}
\includepdf[pages={2-}, scale=0.8]{PDFs/Spezifikationsblatt - Raumbuchungsanzeige - Office365 room booking - censored details.pdf}
\caption{Spezifikationsblatt für die Anwendung}
\label{fig:spezifikationsblatt}
\newline
\newline
%  ~\cite{JavaScript}
%    \printbibliography{title=Literaturverzeichnis}
\end{document}
