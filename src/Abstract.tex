%! Author = mboehme
%! Date = 29.03.2023
\section*{Abstract}\label{sec:Abstract}
Die Problemstellung ist, dass eine ausreichende Ressourcen- und Terminplanung für mehrere Unternehmen oder Abteilungen, innerhalb einer Microsoft Organisation, in den meisten Microsoft Office365 Umgebungen nicht möglich ist.
Basierend auf einem Kundenprojekt wird erläutert werden, inwiefern die Microsoft Graph API v1.0 für solch einen Anwendungsfall sinnvoll ist, wie sie sich im Vergleich zu anderen Lösungen verhält und welche Vorteile sie bietet.
Dabei wird die Microsoft Graph API v1.0 in einer Webseite mit JavaScript implementiert werden, welche Termine verwalten und anzeigen kann.
Um dies sicherzustellen wird anhand eines passenden Modells eine Bewertung der Webseite durchgeführt.
Die Webseite wird auf einem Phillips Display 10BDL4551T/00 Tablet funktionieren und eine eingebaute Lichterkette des Tablets steuern, um den Raumstatus einer Ressource zu visualisieren.
Das Ergebnis ist, dass die Microsoft Graph API v1.0 für solch einen Anwendungsfall sinnvoll ist, da sie eine flexible und erweiterbare Implementierung ermöglicht.
Sie bietet mehr Flexibilität als andere Lösungen, da sie basierend auf individuelle Bedürfnisse genutzt werden kann.
Die Webseite des Kundenprojekts funktioniert auf dem Tablet, kann Termine verwalten und anzeigen und hat zudem die Bedienung vereinfacht und die Sicherheit erhöht.
\newpage